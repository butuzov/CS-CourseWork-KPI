{\descr{Знайти власні значення та вектора матриці}}

$$
A = \begin{pmatrix}
  5 & -1 & -1 \\
  0 & 4 & -1 \\
  0 & -1 & 4
  \end{pmatrix}
$$

$$
\det{A-\lambda E} = \Bigg|\Bigg|\begin{array}{c c c}
  5-\lambda & -1 & -1 \\
  0 & 4-\lambda & -1 \\
  0 & -1 & 4-\lambda
\end{array}\Bigg|\Bigg| = (5-\lambda) (4-\lambda)^2 - (5-\lambda) = (5-\lambda)( (4-\lambda)^2 - 1) = 0.
$$

$$
(5-\lambda)( (4-\lambda)^2 - 1) = 0
$$

Варіанти власних значеннь для $\lambda$ (при яких значення виразу дорівнюватиме нулю).
$$
\begin{array}{ll}
  \lambda_1 = 5 & {\text{для }}  5-\lambda = 0   \\
  \lambda_2 = 5 & {\text{для }}  (4-\lambda)^2=1 \\
  \lambda_3 = 3 & {\text{для }}  (4-\lambda)^2=1 \\
\end{array}
$$


1) При $\lambda = 5$ власним вектором буде

$$
(A-\lambda E)\V{x} =
\begin{bmatrix}
  5-\lambda & -1 & -1 \\
  0 & 4-\lambda & -1 \\
  0 & -1 & 4-\lambda
\end{bmatrix}
\begin{bmatrix} x_1 \\ x_2 \\ x_3 \end{bmatrix}
=
\begin{bmatrix}
  0 & -1 & -1 \\
  0 & -1 & -1 \\
  0 & -1 & -1
\end{bmatrix}
\begin{bmatrix} x_1 \\ x_2 \\ x_3 \end{bmatrix} = 0
$$

$x_2$ основна змінна, $x_3$ - вільна.

$$
\begin{cases}
  -x_2 -x_3 = 0 \\
  -x_2 -x_3 = 0 \\
  -x_2 -x_3 = 0 \\
\end{cases}
\begin{array}{l}\\-\end{array}
\Rightarrow \begin{cases}
  -x_2 -x_3 = 0 \\
  -x_2 -x_3 = 0 \\
\end{cases}
-
\Rightarrow \begin{cases}
  -x_2 -x_3 = 0 \\
\end{cases}  \Rightarrow
\begin{cases}
  -x_2 = x_3 \\
  x_2 = -x_3 \\
\end{cases}
$$


Надаємо вільним членам довільні значення.

$$ x_1=C_1 \qquad x_3=C_3 \qquad
  \begin{cases}
      \V{x} = (C_1;-C_3;C_3) {\text{ або }} (0;-1;1)  \\
      \V{x} = (C_1;-C_3;C_3) {\text{ або }} (1;0;0)  \\
    \end{cases}


$$

\vspace{1cm}

2) При $\lambda_{1,2} = 3$ власним вектором буде

$$
(A-\lambda E)\V{x} =
\begin{bmatrix}
  5-\lambda & -1 & -1 \\
  0 & 4-\lambda & -1 \\
  0 & -1 & 4-\lambda
\end{bmatrix}
\begin{bmatrix} x_1 \\ x_2 \\ x_3 \end{bmatrix}
=
\begin{bmatrix}
  2 & -1 & -1 \\
  0 &  1 & -1 \\
  0 & -1 & 1
\end{bmatrix}
\begin{bmatrix} x_1 \\ x_2 \\ x_3 \end{bmatrix} = 0
$$

$x_1$ та $x_2$ основні змінні, $x_3$ - вільна.

$$
\begin{cases}
  2x_1 -x_2 -x_3 = 0 \\
   x_2 -x_3 = 0 \\
  -x_2 +x_3 = 0 \\
\end{cases}
\begin{array}{l}\\+\end{array}
\Rightarrow \begin{cases}
  2x_1 -x_2 -x_3 = 0 \\
   x_2 -x_3 = 0 \\
\end{cases} \Rightarrow \begin{cases}
x_1 = \dfrac{x_2+x_3}{2} \\
x_2 = x_3 \\
\end{cases}
\Rightarrow \begin{cases}
x_1 = x_3 \\
x_2 = x_3 \\
\end{cases}
$$


Надаємо вільним членам довільні значення.

$$ x_3 = C_3 \qquad \V{x} = (C_3;C_3;C_3)  {\text{ або }} (1;1;1) $$

Власні значення $\boxed{\lambda_{1,2}=5}$ та $\boxed{\lambda_3=3}$
Власні вектора  $ \begin{cases}
      \V{x} =  (C_1;-C_3;C_3) &{\text{ або }} (0;-1;1)  \\
      \V{x} =  (C_1;-C_3;C_3) &{\text{ або }} (1;0;0)  \\
      \V{x} =  (C_3;C_3;C_3)  &{\text{ або }} (1;1;1)  \\
    \end{cases}$
