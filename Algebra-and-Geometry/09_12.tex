{\descr[.3]{Написати канонічне рівняння прямої}}

$$
  x + y + z - 2 = 0 \qquad x-y-2z+2 = 0
$$

Пряма задана у вигляді пересічення двох площин (вектора площиг $\V{A} = \{1;1;1\}$ $\V{B} = \{1;-1;-2\}$), і лежить в обох площинах (тобто перендикулярна обом векторам), звідси вектор $\V{a} = [\V{A}, \V{B}]$ і є канонічним рівнянм прямої.


$$
\V{a} = [\V{A}, \V{B}] = \begin{bmatrix}
    \V{i} & \V{j} & \V{k}  \\
    1     & 1     & 1      \\
    1     & -1    & -2     \\
\end{bmatrix}
=
  i \begin{bmatrix} 1 & 1 \\ -1 & -2 \end{bmatrix}
- j \begin{bmatrix} 1 & 1 \\ 1 & -2 \end{bmatrix}
+ k \begin{bmatrix} 1 & 1 \\ 1 & -1 \end{bmatrix}
= -i +3j -2k
$$

$$
\begin{cases}
  x + y + z - 2 & = 0\\
  x-y-2z+2& = 0
\end{cases}
{\text{ припустимо що }z=0}
\begin{cases}
  x + y- 2 & = 0\\
  x-y+2& = 0
\end{cases} \iff
\begin{cases}
  x + y & = 2  \\
  x - y & = -2 \\
\end{cases} \iff
\begin{cases}
  x & = 0  \\
  y & = 2 \\
\end{cases}
$$

Тобто пряма направлена вздовж вектору $\V{a} = \{-1;3;-2\}$ і проходе через точку $P (0, 2, 0)$, таким чином канонічне рівняння цієї прямої прийме вигляд:

$$
\boxed{ - \dfrac{x-0}{1} = \dfrac{y-2}{3} = - \dfrac{z-0}{2} }
$$
