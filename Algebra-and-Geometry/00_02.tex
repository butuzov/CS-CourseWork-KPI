{\descr[.9]{Знайти обернену матрицю двома методами: за формулою та методом елементарних перетворень. Зробити перевірку}}
\\ \qquad \\
$$
A = \begin{bmatrix}
  1 & 2 &  0 \\
  3 & 0 & -1 \\
  2 & 1 & -1 \\
\end{bmatrix} \qquad \det{A} = -4+6+1 = 3
$$
\center{Знаходження оберненої матриці за методом приєднаної матриці}

$$
\begin{array}{ll}

  \begin{array}{lllll}
    A_{1,1} = & -1^{2} & \begin{bmatrix}0&-1 \\ 1&-1 \end{bmatrix} &= 1
  \end{array}
  &
  \begin{array}{lllll}
    A_{1,2} = & -1^{3} & \begin{bmatrix}3&-1 \\ 2&-1 \end{bmatrix} & = -(-3-(-2)) &= 1 \\
  \end{array}
  \\
  \\
  \begin{array}{lllll}
      A_{1,3} = & -1^{4} & \begin{bmatrix}3&0  \\ 2&1  \end{bmatrix} & = 3
  \end{array}
  &
  \begin{array}{lllll}
    A_{2,1} = & -1^{3} & \begin{bmatrix}2&0  \\ 1&-1 \end{bmatrix} & = -(-2)&=2
  \end{array}
  \\
  \\
  \begin{array}{lllll}
    A_{2,2} = & -1^{4} & \begin{bmatrix}1&0  \\ 2&-1 \end{bmatrix} & = -1
  \end{array}
  &
  \begin{array}{lllll}
    A_{2,3} = & -1^{5} & \begin{bmatrix}1&2  \\ 2&1  \end{bmatrix} & = -(1-4)&=3
  \end{array}
  \\
  \\
  \begin{array}{lllll}
     A_{3,1} = & -1^{4} & \begin{bmatrix}2&0  \\ 0&-1 \end{bmatrix} & = -2
  \end{array}
  &
  \begin{array}{lllll}
    A_{3,2} = & -1^{5} & \begin{bmatrix}1&0  \\ 3&-1 \end{bmatrix} & = -(-1)&=1
  \end{array}
  \\
  \\
  \begin{array}{lllll}
      A_{3,3} = & -1^{6} & \begin{bmatrix}1&2  \\ 3&0  \end{bmatrix} & = -6
  \end{array}
\end{array}
$$



$$
A^{-1} =
\begin{bmatrix}
  \dfrac{A_{1,1}}{\det{A}} & \dfrac{A_{2,1}}{\det{A}} & \dfrac{A_{3,1}}{\det{A}} \\
  \\
  \dfrac{A_{1,2}}{\det{A}} & \dfrac{A_{2,2}}{\det{A}} & \dfrac{A_{3,2}}{\det{A}} \\
  \\
  \dfrac{A_{1,3}}{\det{A}} & \dfrac{A_{2,3}}{\det{A}} & \dfrac{A_{3,3}}{\det{A}} \\
\end{bmatrix}
=
\begin{bmatrix}
  \dfrac{1}{3} & \dfrac{2}{3} & -\dfrac{2}{3} \\
  \\
  \dfrac{1}{3} & -\dfrac{1}{3} & \dfrac{1}{3} \\
  \\
  \dfrac{3}{3} & \dfrac{3}{3} & \dfrac{-6}{3} \\
  \end{bmatrix}
  =
  \begin{bmatrix}
    \dfrac{1}{3} & \dfrac{2}{3} & -\dfrac{2}{3} \\
    \\
    \dfrac{1}{3} & -\dfrac{1}{3} & \dfrac{1}{3} \\
    \\
    1 & 1 & -2 \\
    \end{bmatrix}
$$


\center{Знаходження оберненої матриці методом елементарних перетворень}
$$
\left[
\begin{array}{rrr}
  1 & 2 &  0 \\
  3 & 0 & -1 \\
  2 & 1 & -1 \\
\end{array} \Bigg| \begin{array}{ccc}
  1 & 0 & 0 \\
  0 & 1 & 0 \\
  0 & 0 & 1 \\
\end{array} \right] =
$$

$$
\left[
\begin{array}{rrr}
  1 & 2 &  0 \\
  0 & -6 & -1 \\
  0 & -3 & -1 \\
\end{array} \Bigg|
\begin{array}{ccc}
  _{} \\
  _{-3\cdot{I}} \\
  _{-2\cdot{I}} \\
\end{array}
\Bigg| \begin{array}{rrr}
  1 & 0 & 0 \\
  -3 & 1 & 0 \\
  -2 & 0 & 1 \\
\end{array} \right]= \left[
\begin{array}{rrr}
  1 & 2 &  0 \\
  0 & -6 & -1 \\
  0 & 0 & -\dfrac{1}{2} \\
\end{array} \Bigg|
\begin{array}{ccc}
  _{} \\
  _{} \\
  _{-1/2 \cdot {II}} \\
\end{array}
\Bigg| \begin{array}{rrr}
  1 & 0 & 0 \\
  -3 & 1 & 0 \\
  -\dfrac{1}{2} & -\dfrac{1}{2} & 1 \\
\end{array} \right] =
$$

$$
\left[
\begin{array}{rrr}
  1 & 2 &  0 \\
  0 & -6 & -1 \\
  0 & 0 & 1 \\
\end{array} \Bigg|
\begin{array}{l}
  _{} \\
  _{} \\
  _{\times (-2) } \\
\end{array}
\Bigg| \begin{array}{rrr}
  1 & 0 & 0 \\
 -3 & 1 & 0 \\
  1 & 1 & -2 \\
\end{array} \right] =\left[
\begin{array}{rrr}
  1 & 2 &  0 \\
  0 & -6 & 0 \\
  0 & 0 & 1 \\
\end{array} \Bigg|
\begin{array}{l}
  _{} \\
  _{+ III } \\
  _{} \\
\end{array}
\Bigg| \begin{array}{rrr}
  1 & 0 & 0 \\
  -2 & 2 & -2 \\
  1 & 1 & -2 \\
\end{array} \right] =
$$

$$
\left[
\begin{array}{rrr}
  1 & 2 &  0 \\
  0 & 1 & 0 \\
  0 & 0 & 1 \\
\end{array} \Bigg|
\begin{array}{l}
  _{} \\
  _{/ (-6)} \\
  _{} \\
\end{array}
\Bigg| \begin{array}{rrr}
  1 & 0 & 0 \\
  1/3 & -1/3 & 1/3 \\
  1 & 1 & -2 \\
\end{array} \right] = \left[
\begin{array}{rrr}
  1 & 0 &  0 \\
  0 & 1 & 0 \\
  0 & 0 & 1 \\
\end{array} \Bigg|
\begin{array}{l}
  _{-2 \times II} \\
  _{} \\
  _{} \\
\end{array}
\Bigg| \begin{array}{rrr}
  1/3 & 2/3 & -2/3 \\
  1/3 & -1/3 & 1/3 \\
  1 & 1 & -2 \\
\end{array} \right]
$$
