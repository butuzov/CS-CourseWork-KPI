{\descr{Знайти кут між площинами}}

$$
  3x-y+2z+17=0 \qquad
  5x+9y-3z+1=0
$$

Кут мід площинами - це кут між перпендикулярами до лінії їх перетину, проведеним в цїх площинах. ІНшими словами, в площині $\alpha$ проведемо пряму $a$ перпендикулярну $c$. В площині $\beta$ пряму $b$ перпендикулярну $c$, Кут між площинами $\alpha$ та $\beta$ дорівнєю куту між площинами $a$ і $b$

Векторами площін наразі будуть
$$
\V{A} = \{3;-1;+2\}; \qquad \V{B} = \{5;9;-3\}
$$


$$
  \cos{\widehat{\V{A},\V{B}}}
  = \dfrac{(\V{A},\V{B})}{| \V{A} | \cdot | \V{B} | }
  = \dfrac{3\cdot 5 + -1\cdot 9 + 2\cdot-3 }{\sqrt{3^2+(-1)^2+2^2} \cdot \sqrt{5^2+9^2+(-3)^2} }
  = \dfrac{15-9-6}{\sqrt{14} \cdot \sqrt{115}}
  = \dfrac{0}{\sqrt{1610}}
  = \boxed{0}
$$

Так як $\cos{\widehat{\V{A},\V{B}}}$ дорівнє 0, можемо зробити висновок що кут $\widehat{\V{A},\V{B}}$  дорівнєю $\boxed{90^{\circ}}$
