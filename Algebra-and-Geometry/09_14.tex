{\descr{Найти точку $M'$, симметричную точке $M$ относительно прямой}}

$$
  M \coord{1,2,3} \qquad \dfrac{x-2}{0}=\dfrac{y+1.5}{-1}=\dfrac{z+0.5}{1}
$$

Ми спробуємо знайти точку $M'$ спочатку знайшовши $M_0$ що буде точкою перетину відрізка $MM'$ прямої заданої параметричним рівнянням. Для цього ми спочатку знайдемо перпендикулярну площину що перетинає задані точку і пряму.

Проведемо перепендикулярну площину через точку M і пряму. І так як площина перпендикулярна прямій ми можемо взяти напрявляючий вектор прямої $\{0;-1;1\}$ і в такому разі ми можемо відновити рівняння перпендикулярної площини

$$
0(x-1) - 1(y-2) + 1(z-3) = -y+z+2-3 = -y+z-1 = 0
$$

Наступним кроком - ми спробуємо знайти точку $M_0$ точку перетину прямої та площини.

Запишемо параметричне рівнння прямої

$$
\dfrac{x-0.5}{0} = \dfrac{y+1.5}{-1} = \dfrac{z-1.5}{1} = t \iff
\begin{cases}
  x = 0.5 \\
  y = -1.5 - t \\
  z = 1.5+t
  \end{cases}
$$

і підставимо в рівняння площини:

$$
  -(-1.5-t)+(1.5+t)-1 = 1.5+t+1.5+t-1 = 3+2t-2 = 2+2t = 0 \Rightarrow \boxed{t = -1}
$$

Координати $M_0$ таким чином будуть дорівнювати

$$
  \begin{cases}
    x = 0.5 \\
    y = -1.5 - ( -1 ) = -0.5 \\
    z = 1.5 + ( -1 ) = 0.5
    \end{cases} \Rightarrow M_0 (0.5;-0.5;0.5)
$$

Так як $M_0$ це середина відрізку $MM'$

$$
\begin{cases}
  x_{M_0} = \dfrac{x_{M'}+x_M}{2} \\
  \\
  y_{M_0} = \dfrac{y_{M'}+y_M}{2} \\
  \\
  z_{M_0} = \dfrac{z_{M'}+z_M}{2} \\
\end{cases} \iff
\begin{cases}
  x_{M'} = 2 \cdot x_{M_0} - x_M \\
  \\
  y_{M'} = 2 \cdot y_{M_0} - y_M \\
  \\
  z_{M'} = 2 \cdot z_{M_0} - z_M \\
\end{cases} =
\begin{cases}
  x_{M'} = 2 \cdot 0.5 - 1 \\
  \\
  y_{M'} = 2 \cdot (-0.5) - 2 \\
  \\
  z_{M'} = 2 \cdot 0.5 - 3 \\
\end{cases}=
\begin{cases}
  x_{M'} = 0 \\
  \\
  y_{M'} = -3 \\
  \\
  z_{M'} = -2 \\
\end{cases}
$$

$$
\boxed{M'=(0;-3;-2)}
$$
