{\descr[2]{Написати розлад вектора $x$  по векторам $p$, $q$, $r$ , тобто, знайти такі $\alpha$, $\beta$, $\gamma$ що  $ \alpha p+ \beta q+\gamma r=x$.  Отриману систему розв’язати методом оберненої матриці, методом приєднаної матриці та за формулами Крамера.
}}

$$
  x =\{\cord{-9,5,5}\} \qquad  p=\{\cord{4,1,1}\} \qquad q=\{\cord{2,0,-3}\}  \qquad r=\{\cord{-1,2,1}\}
$$

\begin{center}$ \alpha p+ \beta q+\gamma r=x$ може бути поданоу вигляді системи лінійних рівнянь\end{center}

$$
\begin{cases}
    4\alpha + 1\beta + 1\gamma = -9 \\
    2\alpha - 3\gamma = 5 \\
    -1\alpha + 2\beta + 1\gamma = 5
\end{cases}
$$

Спробуйємо розвязати систему рівнянь трьомарізними способами.


\begin{center}\textbf{ Метод оберненої матриці} \end{center}

  Метод полягає у знайденні оберненої матриці і множенні оберненої матриці на матрицю вільних членів, це допомогає знайти совбчик невідомих. Зазвичай використовують метод приєднаної матриці щоб знайти обернену, але так як це окреме завдання - знайдемо обернену матрицю методом Гауса.
%
$$
    A = \begin{bmatrix}
      4 & 1 &  1 \\
      2 & 0 & -3 \\
     -1 & 2 &  1 \\
    \end{bmatrix} \qquad \det{A} = 3 + 4 - 2 - (- 32) = 29
$$

$$
  \left[
  \begin{array}{rrr}
    4 & 1 &  1 \\
    2 & 0 & -3 \\
    -1 & 2 & -1 \\
  \end{array} \Bigg| \begin{array}{ccc}
    1 & 0 & 0 \\
    0 & 1 & 0 \\
    0 & 0 & 1 \\
  \end{array} \right] =
$$

$$
  \left[
  \begin{array}{rrr}
   -1 & 2 & 1 \\
    2 & 0 & -3 \\
    4 & 1 &  1 \\
  \end{array} \Bigg|
  \begin{array}{ccc}
    _{III} \\
    _{} \\
    _{I} \\
  \end{array}
  \Bigg| \begin{array}{rrr}
    0 & 0 & 1 \\
    0 & 1 & 0 \\
    1 & 0 & 0 \\
  \end{array} \right]= \left[
  \begin{array}{rrr}
   -1 & 2 & 1 \\
    0 & 4 & -1 \\
    4 & 1 &  1 \\
  \end{array} \Bigg|
  \begin{array}{ccc}
    _{} \\
    _{+2 \cdot I} \\
    _{} \\
  \end{array}
  \Bigg| \begin{array}{rrr}
    0 & 0 & 1 \\
    0 & 1 & 2 \\
    1 & 0 & 0 \\
  \end{array} \right] =
$$

$$
  \left[
  \begin{array}{rrr}
  -1 & 2 & 1 \\
    0 & 4 & -1 \\
    0 & 9 &  5 \\
  \end{array} \Bigg|
  \begin{array}{ccc}
    _{} \\
    _{} \\
    _{+4 \cdot I} \\
  \end{array}
  \Bigg| \begin{array}{rrr}
    0 & 0 & 1 \\
    0 & 1 & 2 \\
    1 & 0 & 4 \\
  \end{array} \right] = \left[
  \begin{array}{rrr}
  -1 & 2 & 1 \\
    0 & 9 &  5 \\
    0 & 4 & -1 \\
  \end{array} \Bigg|
  \begin{array}{ccc}
    _{} \\
    _{III} \\
    _{II} \\
  \end{array}
  \Bigg| \begin{array}{rrr}
    0 & 0 & 1 \\
    1 & 0 & 4 \\
    0 & 1 & 2 \\
  \end{array} \right] =
$$

$$
\left[
\begin{array}{rrr}
-1 & 2 & 1 \\
  0 & 9 &  5 \\
  0 & 0 & -29/9 \\
\end{array} \Bigg|
\begin{array}{ccc}
  _{} \\
  _{} \\
  _{-(4\cdot{II})/9} \\
\end{array}
\Bigg| \begin{array}{rrr}
  0 & 0 & 1 \\
  1 & 0 & 4 \\
  -4/9 & 1 & 2/9 \\
\end{array} \right] = \left[
\begin{array}{rrr}
-1 & 2 & 1 \\
  0 & 9 &  5 \\
  0 & 0 & 1 \\
\end{array} \Bigg|
\begin{array}{ccc}
  _{} \\
  _{} \\
  _{\cdot -9/29} \\
\end{array}
\Bigg| \begin{array}{rrr}
  0 & 0 & 1 \\
  1 & 0 & 4 \\
  4/29 & -9/29 & -2/29 \\
\end{array} \right]
$$

$$
\left[
\begin{array}{rrr}
-1 & 2 & 0 \\
  0 & 9 &  0 \\
  0 & 0 & 1 \\
\end{array} \Bigg|
\begin{array}{ccc}
  _{-III} \\
  _{-5\cdot{III}} \\
  _{} \\
\end{array}
\Bigg| \begin{array}{rrr}
  -4/29 & 9/29 & 31/29 \\
  9/29 & 45/29 & 126/29 \\
  4/29 & -9/29 & -2/29 \\
\end{array} \right] = \left[
\begin{array}{rrr}
-1 & 2 & 0 \\
  0 & 1 &  0 \\
  0 & 0 & 1 \\
\end{array} \Bigg|
\begin{array}{ccc}
  _{} \\
  _{/9} \\
  _{} \\
\end{array}
\Bigg| \begin{array}{rrr}
  -4/29 & 9/29 & 31/29 \\
  1/29 & 5/29 & 14/29 \\
  4/29 & -9/29 & -2/29 \\
\end{array} \right]
$$

$$
\left[
\begin{array}{rrr}
-1 & 0 & 0 \\
  0 & 1 &  0 \\
  0 & 0 & 1 \\
\end{array} \Bigg|
\begin{array}{ccc}
  _{-2\cdot{II}} \\
  _{} \\
  _{} \\
\end{array}
\Bigg| \begin{array}{rrr}
  -6/29 & -1/29 & 3/29 \\
  1/29 & 5/29 & 14/29 \\
  4/29 & -9/29 & -2/29 \\
\end{array} \right] = \left[
\begin{array}{rrr}
 1 & 0 & 0 \\
  0 & 1 &  0 \\
  0 & 0 & 1 \\
\end{array} \Bigg|
\begin{array}{ccc}
  _{\cdot{-1}} \\
  _{} \\
  _{} \\
\end{array}
\Bigg| \begin{array}{rrr}
  6/29 & 1/29 & -3/29 \\
  1/29 & 5/29 & 14/29 \\
  4/29 & -9/29 & -2/29 \\
\end{array} \right]
$$

$$
A = \begin{bmatrix}
  4 & 1 &  1 \\
  2 & 0 & -3 \\
 -1 & 2 &  1 \\
\end{bmatrix}^{-1} = \dfrac{1}{29}  \begin{bmatrix}
  6 & 1 & -3 \\
  1 & 5 & 14 \\
  4 & -9 & -2 \\
\end{bmatrix}
$$

Після знаходження оперненої матриці можна переходити до її множення на стопчик вільних коофіцієнтів щоб отримати стовпчик невідомих.

$$
X = \begin{bmatrix}
  \alpha \\
  \beta  \\
  \gamma \\
\end{bmatrix} =
A^{-1} \cdot B =
\dfrac{1}{29} \begin{bmatrix}
  6 & 1 & -3 \\
  1 & 5 & 14 \\
  4 & -9 & -2 \\
\end{bmatrix} \times \begin{bmatrix}
  -9 \\
  5  \\
  5 \\
\end{bmatrix} = \dfrac{1}{29} \begin{bmatrix}
-54+5-15\\
\\
-9+25+70\\
\\
-36-45-10\\
\end{bmatrix} = \begin{bmatrix}
-\dfrac{64}{29} \\
\\
\dfrac{86}{29} \\
\\
- \dfrac{91}{29} \\
\end{bmatrix}
$$


\begin{center}\textbf{ Метод приєднаної матриці } \end{center}

  $$
    A = \begin{bmatrix}
      4 & 1 &  1 \\
      2 & 0 & -3 \\
     -1 & 2 &  1 \\
    \end{bmatrix} \qquad \det{A} = 3 + 4 - 2 - (- 32) = 29
  $$

Метод приєднаної матриці являє собою майже повний аналог методу оберненої матриці, але обернена матриця виражається через - приєднану розділену на детермінант.

$$
A^* =
\begin{bmatrix}
  A_{11} & A_{12} & A_{13} \\
  A_{21} & A_{22} & A_{23} \\
  A_{31} & A_{32} & A_{33} \\
\end{bmatrix}
\qquad
X = \begin{bmatrix}
  \alpha \\
  \beta  \\
  \gamma \\
\end{bmatrix}
\qquad
B = \begin{bmatrix}
  -9 \\
  5  \\
  5 \\
\end{bmatrix}
$$

$$
\begin{array}{ll}

  \begin{array}{lllll}
      A_{11} = & (-1)^{1+1} & \begin{bmatrix}0&-3\\2&1 \end{bmatrix} & =  6
  \end{array}
  &
  \begin{array}{lllll}
    A_{12} = & (-1)^{1+2} & \begin{bmatrix}2&-3\\-1&1 \end{bmatrix} & = -(2-3) = 1
  \end{array}
  \\
  \\
  \begin{array}{lllll}
      A_{13} = & (-1)^{1+3} & \begin{bmatrix}2&0\\-1&2 \end{bmatrix} & = 4
  \end{array}
  &
  \begin{array}{lllll}
    A_{21} = & (-1)^{2+1} & \begin{bmatrix}1&1\\2&1 \end{bmatrix} & = -1(1-2) = 1
  \end{array}
  \\
  \\
  \begin{array}{lllll}
    A_{22} = & (-1)^{2+2} & \begin{bmatrix}4&1\\-1&1 \end{bmatrix} & =  4+1  = 5
  \end{array}
  &
  \begin{array}{lllll}
    A_{23} = & (-1)^{2+3} & \begin{bmatrix}4&1\\-1&2 \end{bmatrix} & = -1(8+1) = -9
  \end{array}
  \\
  \\
  \begin{array}{lllll}
     A_{31} = & (-1)^{3+1} & \begin{bmatrix}1&1\\0&-3 \end{bmatrix} & = -3
  \end{array}
  &
  \begin{array}{lllll}
    A_{32} = & (-1)^{3+2} & \begin{bmatrix}4&1\\2&-3 \end{bmatrix} & =-1(-12-2) = 14
  \end{array}
  \\
  \\
  \begin{array}{lllll}
      A_{33} = & (-1)^{3+3} & \begin{bmatrix}4&1\\2&0 \end{bmatrix} & = -2
  \end{array}
\end{array}
$$


Після знаходження оперненої матриці можна переходити до її множення на стопчик вільних коофіцієнтів щоб отримати стовпчик невідомих.


$$
A^{-1} = \dfrac{1}{\det{A}} \cdot  A^{*T} =
\dfrac{1}{\det{A}} \begin{bmatrix}
  A_{11} & A_{21} & A_{31} \\
  A_{12} & A_{22} & A_{32} \\
  A_{13} & A_{23} & A_{32} \\
\end{bmatrix} =
\dfrac{1}{29} \begin{bmatrix}
  6 & 1 & -3 \\
  1 & 5 & 14 \\
  4 & -9 & -2 \\
\end{bmatrix}
$$

$$
X = \begin{bmatrix}
  \alpha \\
  \beta  \\
  \gamma \\
\end{bmatrix} =
A^{-1} \cdot B =
\dfrac{1}{29} \begin{bmatrix}
  6 & 1 & -3 \\
  1 & 5 & 14 \\
  4 & -9 & -2 \\
\end{bmatrix} \times \begin{bmatrix}
  -9 \\
  5  \\
  5 \\
\end{bmatrix} = \dfrac{1}{29} \begin{bmatrix}
-54+5-15\\
\\
-9+25+70\\
\\
-36-45-10\\
\end{bmatrix} = \begin{bmatrix}
-\dfrac{64}{29} \\
\\
\dfrac{86}{29} \\
\\
- \dfrac{91}{29} \\
\end{bmatrix}
$$

$$
\boxed{X = \begin{bmatrix}
  \alpha \\
  \beta  \\
  \gamma \\
\end{bmatrix} = \begin{bmatrix}
-\dfrac{64}{29} \\
\\
\dfrac{86}{29} \\
\\
- \dfrac{91}{29} \\
\end{bmatrix} }}
$$

\begin{center}\textbf{ Метод Крамера }\end{center}

Метод Крамера полягає у визначенні детермінантів окремо для матриць де шуканий стовбичк підміняється результатом.

$$
  A = \begin{bmatrix}
    4 & 1 &  1 \\
    2 & 0 & -3 \\
   -1 & 2 &  1 \\
  \end{bmatrix} \qquad \det{A} = 3 + 4 - 2 - (- 32) = 29
$$

$$
\det{\alpha} = \begin{bmatrix}
  s_1 & q_1 & r_1 \\
  s_2 & q_2 & r_1 \\
  s_3 & q_3 & r_1 \\
\end{bmatrix} = \begin{bmatrix}
  -9 & 1 &  1 \\
   5 & 0 & -3 \\
   5 & 2 &  1 \\
\end{bmatrix} = -64
$$

$$
\det{\beta} = \begin{bmatrix}
  p_1 & s_1 & r_1 \\
  p_2 & s_2 & r_1 \\
  p_3 & s_3 & r_1 \\
\end{bmatrix} = \begin{bmatrix}
  4 & -9 &  1 \\
  2 & 5 & -3 \\
  -1 & 5 &  1 \\
\end{bmatrix} = 86
$$

$$
\det{\gamma} = \begin{bmatrix}
  p_1 & q_1 & s_1 \\
  p_2 & q_2 & s_1 \\
  p_3 & q_3 & s_1 \\
\end{bmatrix} = \begin{bmatrix}
   4 & 1 & -9 \\
   2 & 0 & 5  \\
  -1 & 2 & 5  \\
\end{bmatrix} = -91
$$

$$
\boxed{
  \alpha = \dfrac{\det{\alpha}}{\det{A}} = -\dfrac{64}{29} \qquad \beta = \dfrac{\det{\beta}}{\det{A}} = \dfrac{86}{29} \qquad \gamma = \dfrac{\det{\gamma}}{\det{A}} = -\dfrac{91}{29}
}
$$
