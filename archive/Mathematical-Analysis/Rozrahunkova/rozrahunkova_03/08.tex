{\descr{Знайдіть наближено суму ряду з точністю \varepsilon}}

$$
  \sum_{n=1}^\infty \dfrac{(-1)^{n+1}}{(n+2)^{2n}}, \qquad \varepsilon = 0,001
$$

\begin{enumerate}
  \item  Знайдемо чому дорівнює границя:
         $$ \lim_{n \to \infty } \dfrac{1}{(n+2)^{2n}} \approx 0 $$

 \item Спробуємо використати ознаку Діріхлє для доведення збіжності знакозмінного ряду:

  $$
    \lim_{n \to \infty} \sqrt[n]{(\dfrac{1}{n+2})^{2n}}
  = \lim_{n \to \infty} \dfrac{1}{(n+2)^2}
  = \dfrac{1}{(\infty+2)^2}\approx 0 \Rightarrow {\text{ 0 < 1, тобто ряд збіжний}}
  $$

  \item Перевіримо ряд на ознаки ряду Лейбніца:
    \begin{itemize}
        \item Ряд задовільняє першу умову оскільки ряд є спадаючим (починаючи з n > 1).
        \item Як вже визначено в пункті 1. границя $u_n$ приблизно дорівнює нулю.
      \end{itemize}

\end{enumerate}

З чого можна зробити висновок що данний ряд є рядом Лейбніца.

Абсолютна похибка від заміни його суми ряду його n-ою чистиною сумою не перевищує модуля першого з його членів, що видкривається:

$$
\Big| r_n \Big| = \Big| S-S_n \Big| \leqslant \Big| u_{n+1} \Big| \leqslant \varepsilon
$$

\M{Підбираємо значення $n$, для якого виконується рівння}

$$\Big( \dfrac{1}{((n+1)+2)^{2n+1}} \Big) \leqslant 0.001 \Rightarrow n = \Boxed{2} \rightarrow \dfrac{1}{((2+1)+2)^2(2\cdot(2+1))} < 0.001 $$

Тому сума перших двох членів ряду дасть нам суму ряду з точністю $\varepsilon = 0.001$ :

$$
  S \approx = \dfrac{1}{3^2} + \dfrac{1}{4^4} = \boxed{0.1151}
$$
