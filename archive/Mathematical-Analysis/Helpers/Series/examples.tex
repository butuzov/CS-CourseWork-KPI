\M{\large{\textbf{Приклади}}}

$$
1) \sum_{n=1}^{\infty} \dfrac{1}{2^n+1} \qquad \Bigg| a_n = \dfrac{1}{2^n+1} < \dfrac{1}{2^n} \Bigg|
$$

\M{Ряд \textbf{є збіжним за першою ознакою ()} - оскільки загальний член ряду менше загального  загального члену $ (1/2)^n$ який є з.ч.  спадаючої геометричної прогресії, а вона є збіжним числовим рядом.}

$$
2) \sum_{n=1}^{\infty} \dfrac{1}{3n+3}
  \qquad
  \Bigg|
    \begin{array}{rlrl}
      a_n = & \dfrac{1}{3n+3} & b_n = & \dfrac{1}{n} \\
      \end{array}
      \Bigg| \Rightarrow
    \lim_{n \to \infty} \dfrac{a_n}{b_n} = \lim_{n \to \infty} \dfrac{n}{3n+3} = \dfrac{1}{3} \lim_{n \to \infty} \dfrac{1}{1+1/n} = \dfrac{1}{3}
$$

\M{Ряд є \textbf{розбіжним за другою ознакою збіжності} - оскільки границею від частки загального елементу нашого ряду на загальний елемент гармонічного ряду є кінцеве і відмінне від нуля число, а значить наш ряд має туж характеристику що і ряд щагальний елемент якого ми вкористали для діленя (тобто "розбіжний").}

$$
3) \sum_{n=0}^\infty \Big( \dfrac{n}{2n+1} \Big)^n
\Rightarrow
 \lim_{n \to \infty} \sqrt[n]{\Big( \dfrac{n}{2n+1} \Big)^n}
= \lim_{n \to \infty} \dfrac{n}{2n+1}
= \dfrac{1}{2}
$$

\M{Ряд є збіжним за \textbf{ознакою Коші}}

$$
4) \sum_{n=0}^\infty \dfrac{2^n}{n^{10}}
  \Rightarrow
  \lim_{n \to \infty} \dfrac{2^{n+1}}{(n+1)^{10}} \times \dfrac{n^{10}}{2^n}
=  \lim_{n \to \infty} \dfrac{2n^{10} }{(n+1)^{10} }
=  2 \lim_{n \to \infty} \Big( \dfrac{n}{n+1} \Big)^{10}
= 2
$$

\M{Ряд є розбіжним за \textbf{ознакою Даламбера}}

$$
5) \sum_{n=1}^\infty \dfrac{1}{n^2}
\Rightarrow
    \int_{1}^\infty \dfrac{1}{x^2} \d{x}
=  - \dfrac{1}{x} \Bigg|_{1}^\infty
=  - (\dfrac{1}{\infty} - \dfrac{1}{1})
=  0 + \dfrac{1}{1} = 1
$$


\M{Ряд є збіжним за інтегральною ознакою}



$$
6) \sum_{n=1}^\infty -1^{n+1}\dfrac{1}{n} = 1 - \dfrac{1}{2} + \dfrac{1}{3} - \dfrac{1}{4} \ldots + (-1)^{n+1} \dfrac{1}{n}
$$

Ряд знакозмінний, шукаємо ознаки Лейбніца, обидві умови для ряду Лейбніца виконуються, але ряд за модулем є розбіжним (за порівняльною ознакою, де ми порівнбємо з членом гармонічного ряду) тобто ряд є умовно збіжним.
