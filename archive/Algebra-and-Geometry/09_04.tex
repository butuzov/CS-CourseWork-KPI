{\descr[.9]{Знайдість площу паралелограмма, побудованого на векторах $a$ та $b$.
}}

$$
  \V{a} = 3\V{p}-2\V{q} \qquad \V{b}=\V{p}+5\V{q}}; \qquad |\V{p}| = 4, |\V{q}| = \dfrac{1}{2} \quad \sin{\widehat{(p,q)}}=\dfrac{5\pi}{6}
$$

Довжина векторного добутку двох векторів чисельно дорівнює площі паралелограма, який побудований на векторах-співмножниках відкладених від спільної точки.

Спочатку знайдемо векторний добуток вектора $a$ на $b$.

$$
  [\V{a};\V{b}]
= [3\V{p}-2\V{q}, \V{p}+5\V{q}]
= 3[\V{p}, \V{p}] + 15[\V{p}, \V{q}] - 2[\V{q}, \V{p}] - 10 [\V{q},\V{q}]
= 17[\V{p}, \V{q}]
$$

Маючи векорний добуток, знайдемо площу.

$$
S = |a \cdot b|
  = | 17 \cdot \V{p} \cdot \V{q} |
  =   17 \cdot | \V{p} \cdot \V{q} |
  =   17 \cdot \V{|p|} \cdot \V{|q|} \cdot \sin{\widehat{(p,q)}}
  = 17 \cdot 4 \cdot \dfrac{1}{2} \cdot \sin{\dfrac{5\pi}{6}}
  = 34 \cdot \dfrac{1}{2} = 17
$$

Площа паралелограмма  побудованого на векторах $a$ та $b$ - 17
