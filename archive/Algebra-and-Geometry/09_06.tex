{\descr[1]{Визначита обєм тетраєдра з вершинами в точках $A_1,A_2,A_3, A_4$ і його висоту, спущену з вершини $A_4$ на грань $A_1,A_2,A_3$.
}}

$$
A_1 = \coord{2,   1,  4} \qquad
A_2 = \coord{-1,  5, -2} \qquad
A_3 = \coord{-7, -3,  2} \qquad
A_4 = \coord{-6, -3,  6}
$$


Проведемо вектори з вершини A_1

$$
\begin{array}{lrlrlrlrcrcr}
  \V{A_1A_2} = & \{-1 -2&;& 5-1&;& -2-4 &\}&= \{ -3 &;& 4 &;& -6 \}\\
  \V{A_1A_3} = & \{-7 -2&;&-3-1&;&  2-4 &\}&= \{ -9 &;&-4 &;& -2\}\\
  \V{A_1A_4} = & \{-6 -2&;&-3-1&;&  6-4 &\}&= \{ -8 &;& -4 &;& 2 \}\\
  \end{array}
$$

Формула за якою ми знайдемо об'єм піраміди:

$$
V_{A_1A_2A_3A_4} = \dfrac{1}{6}\Big|\Big(\V{A_1A_2},\V{A_1A_3},\V{A_1A_4} \Big)\Big|
$$

Отож занйдемо, змішаний добуток
$$
 \Big(\V{A_1A_2},\V{A_1A_3},\V{A_1A_4} \Big)
% \begin{array}{c}
%   \V{A_1A_2} \\
%   \V{A_1A_3} \\
%   \V{A_1A_4}
% \end{array}
\begin{bmatrix}
  -3 &  4 &  -6 \\
  -9 &  -4 &  -2 \\
  -8 &  -4 &  2  \\
\end{bmatrix} =
\begin{array}{l}
%  (-3 \cdot -4\cdot  2 )+( 4 \cdot -2 \cdot -8 )+(-9 \cdot -4 \cdot -6 )
% -(-6 \cdot -4 \cdot  -8 )-(-9 \cdot 4 \cdot 2)-(-4\cdot  -2\cdot  -3) \\
24+64-216+192+72+24 = 160
\end{array}
$$

І обєм піраміди завдяки змішаному добутку.

$$
\Big(\V{A_1A_2},\V{A_1A_3},\V{A_1A_4} \Big) = \dfrac{1}{6} \cdot 160  = \dfrac{80}{3} 
$$

Висоту піраміди ми можемо обчислити завдяки іншій формулі обєму що виглядає наступним чином

$$
V = \dfrac{1}{3}Sh \iff h = \dfrac{3V}{S}
$$
де $h$ висота спущена з вершини $A_4$  на грань $A_1,A_2,A_3$

$$
S_{A_1A_2A_3} = \dfrac{ | \V{A_1A_2},\V{A_1A_3} | }{2}
$$

Векторний добуток:
$$
\begin{bmatrix}
  i &  j &  k \\
  -3 &  4 &  -6 \\
  -9 &  -4 &  -2 \\
\end{bmatrix} =
i \cdot \begin{bmatrix}
    4 &  -6 \\
   -4 &  -2 \\
\end{bmatrix} -
j \cdot \begin{bmatrix}

  -3 &     -6 \\
  -9 &     -2 \\
\end{bmatrix} +
k \cdot \begin{bmatrix}
  -3 &  4   \\
  -9 &  -4   \\
\end{bmatrix} = -i32 + j48 + k 48 = \{-32;48;48\}
$$

$$
S_{A_1A_2A_3} = \dfrac{ \sqrt{-32^2+48^2+48^2} }{2} = \dfrac{ \sqrt{5632} }{2} = = \sqrt{\dfrac{5632}{4}} = \sqrt{1408} = \sqrt{ 64 \cdot 22} = 8 \sqrt{22}
$$

$$
h = \dfrac{3 \cdot \dfrac{80}{3}}{8\sqrt{22}}
= \dfrac{80}{8\sqrt{22}}
= \dfrac{10}{\sqrt{22}}
$$
$$
\boxed{V=\dfrac{80}{3} \qquad h = \dfrac{10}{\sqrt{22}}}
$$
