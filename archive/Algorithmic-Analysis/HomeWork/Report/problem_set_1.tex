\begin{center}
  \Large{\cyr{\textbf{Завдання №1}}}
\end{center}

Спроектувати алгоритм розв’язку задачі використовуючи базові алгоритмічні конструкції, реалізувати мовою програмування С/С++. У звіті вказати, яким вимогам повинен задовольняти розв’язок задачі, вхідні дані та результати, представити блок-схему алгоритму (або псевдокод, або діаграму дії, тощо), текст програмного коду.

% \begin{enumerate}
% \item Опрацювати теоретичний матеріал та літературні джерела до теми №1.
% \item Спроектувати алгоритм розв’язку задачі використовуючи базові алгоритмічні конструкції, реалізувати мовою програмування С/С++. У звіті вказати, яким вимогам повинен задовольняти розв’язок задачі, вхідні дані та результати, представити блок-схему алгоритму (або псевдокод, або діаграму дії, тощо), текст програмного коду.
% \end{enumerate}

$$
 y = \left\{
    \begin{array}{lcr}
       (x^2+5x)(\dfrac{1}{x}+\sqrt{x+4}), & \text{якщо} & x < 1 \\
       \\
       \sum_{m=1}^7 \dfrac{m-x}{\sqrt{m+x}} & \text{якщо} & x \geq 1 \\
    \end{array} \begin{array}{l}
    x \in [-1,5;3,5] \\
    {\text{ крок }} h=0.25
    \end{array}
$$


За умовами задачі створено алгоритм (дивіться представлення алгоритму на насутпній сторінці) та реалізацію программи мовою C++. Данна программа компілюється і запускається з вихідного файлу $problem\_1.cpp$ наступним чином:

$\normalsize\texttt{c++ problem\_1.cpp -o algorithm -std=c++11 \&\& ./algorithm}$
