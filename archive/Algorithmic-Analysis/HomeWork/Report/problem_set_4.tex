\begin{center}
  \Large{\cyr{\textbf{Завдання №4}}}
\end{center}

\begin{quote} \textbf{Завдання}: У впорядкованому масиві цілих чисел ai , i=1...n знайти номер елемента c використовуючи метод двійкового пошуку. Передбачається, що цей елемент знаходиться в масиві.
\end{quote}

Алгоритм бінарного пошуку, являє собою швидкий ( $O(\log{n})$ ) алгоритм пошуку, мінусом данного алгоритму є те що працює він лише на попередньо відсортованих масивах. Алгоритм роботи полягає в наступному:
\begin{enumerate}
  \item Визначаються початковий і останній елемент масива для пошука.
  \item Починаємо цикл з умовю що допоки початок не дорівнюватиме кінцю ми будемо ітерувати тіло циклу.
  \item Визначимо центр пошукового діапазону, додавши що "початку" - середнє різниці між кінцем та початком.
  \item Якщо Елемент масиву дорівнює тому значенню що ми шукаємо, робота алгоритму повертається і ми повертає індекс знайденого елементу, або логічну змінну що елемент знайдено.
  \item Якщо Елемент масиву менше того значення що ми шукаємо, логічно уявити що  шукане значення знаходить в правій стороні, в такому разі ``початком'' діапазону пошуку стає наступний за порівнювальним елемент масиву ( middle+1 ).
  \item Якщо Елемент масиву більше того значення що ми шукаємо, логічно уявити що  шукане значення знаходить в лівій стороні, в такому разі ``кінцем'' діапазону пошуку стає попередній порівнювальним елемент масиву ( middle ).
  \item Якщо цикл закінчився без результатно то повертається значення -1, що означає щ індекс не знайдено.
\end{enumerate}


Імплементація розташована в  $\normalsize\texttt{code.cpp}$, реалізацію і приклад використання в $\normalsize\texttt{problem\_4\_binary\_search.cpp}$, данний приклад можна скомпілювати за запустит за допомогою команд оболонки bash \\  $\normalsize\texttt{c++ problem\_4\_binary\_search.cpp code.cpp -o bs -std=c++11 \&\& ./bs}$
