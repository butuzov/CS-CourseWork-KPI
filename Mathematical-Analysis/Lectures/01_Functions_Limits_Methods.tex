
\begin{center}\large{\cyr{\textbf{Обчислення границь методом підстановки}}}\end{center}

За цим методом ми просто \textbf{підставляємо} граничне значення.

\begin{displaymath}
\lim_{n \to 3} \frac{(x^3+3x)}{2x+5} = \frac{9+9}{6+5} = \frac{18}{11} = 1 \end{displaymath}

Підставили значення, обчислили, записали границю у відповідь. Зате на базі таких границь можемо привчитись, що перш за все потрібно підставити значення у функцію ( для перевірки відповіді на питання - 'що отримаємо?' чи 'до чого прямує'?).

\begin{center}\large{\cyr{\textbf{Невизначеність}}}\end{center}

\begin{displaymath}
  \lim_{n \to \infty} \frac {(x^2+2x)}{4x^2+3x-4}
    = \Bigg[ \frac{\infty}{\infty} \Bigg]
    = \lim_{n \to \infty} \frac{x^2(1+\frac{2}{x})}{x^2(4+\frac{3}{x}-\frac{4}{x^2})}
    = \lim_{n \to \infty} \frac{1}{4}
\end{displaymath}

Алгоритм обчислення границі полягає у знаходженні \textbf{найбільшого степеня "x"} в чисельнику чи знаменнику. Далі на нього ділять чисельник і знаменник і знаходять границю. Відразу з практики можна отримати два висновки, які є підказкою в обчисленнях:
\begin{description}
  \item Якщо змінна прямує до безмежності і степінь чисельника більший від степені знаменника то границя рівна безмежності.
  \item В протилежному випадку, якщо поліном в знаменнику старшого порядку ніж в чисельнику границя рівна нулю.
\end{description}

\begin{displaymath}
  \lim_{n \to \infty} \frac{ax^n+\dots}{bx^n+\dots} =
  \begin{cases}
    \frac{a}{b}, n=m \\
    \infty, n > m \\
    0, n < m
  \end{cases}
\end
