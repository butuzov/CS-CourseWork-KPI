\begin{center}\large{\cyr{\textbf{Границя функції в точці}}}\end{center}

\begin{quote}

\textbf{Границя функції в точці} — фундаментальне поняття математичного аналізу, зокрема аналізу функцій дійсної змінної, число, до якого прямує значення функції, якщо її аргумент прямує до заданої точки. Строге математичне означення границі функції дається мовою $\delta-\varepsilon$.

\textbf{Одностороння границя} — це границя функції однієї змінної в деякій точці, коли аргумент прямує до значення аргументу у цій точці окремо зі сторони більших аргументів (\textbf{правостороння границя}), або зі сторони менших аргументів (\textbf{лівостороння границя}). Тобто, є сенс говорити про односторонні границі функції у деякій точці тільки тоді, коли у цій точці \textbf{лівостороння границя функції не дорівнює правосторонній}.
\end{quote}


\begin{description}

  \item[Лівосторонню границю] прийнято позначати:
    \begin{displaymath} \lim_{x \to x_{0-}}=A \qquad
        \forall\varepsilon > 0, \exists \delta > 0, \qquad
        \forall{x}: x_0-\delta < x < x_0 \end{displaymath}

  \item[Правосторонню границю] прийнято позначати:
    \begin{displaymath} \lim_{x \to x_{0+}}=B \qquad
        \forall\varepsilon > 0, \exists \delta > 0, \qquad
        \forall{x}: x_0 < x < x_0 + \delta \end{displaymath}

\end{description}


Функція неперерва в точці $x=a$ якщо вона:
\begin{enumerate}
  \item Вона визначена в цій точці
  \item Існує границя функції в цій точці
    \begin{displaymath}\lim_{x \to a} f(x)\end{displaymath}
  \item Значення границі дорівнює значенню функції в точці
    \begin{displaymath}\lim_{x \to a} f(x) = f(a)\end{displaymath}
\end{enumerate}


\begin{quote}Якщо одна із умов порушується, то функція називається \textbf{розривною в точці} х=а, а сама точка х=а називається \textbf{точкою розриву}. Усі елементарні функції є неперервними на інтервалах визначеності.\end{quote}

Точка $х_0$ називається \textbf{точкою розриву першого роду} функції $у = f(x)$, якщо існують скінчені односторонні границі зліва та зправа (константи):

\begin{displaymath} \lim_{x \to x_{0+}}=C_1 \qquad \lim_{x \to x_{0-}}=C_2 \end{displaymath}

Функція в точці x=a \textbf{має неусувний розрив першого роду}, за умов:

  \begin{displaymath}
    \begin{cases}
      \lim_{x \to a_{0+}}f(x) \neq f(a) \\
      \lim_{x \to a_{0-}}f(x) \neq f(a) \\
      \lim_{x \to a_{0-}}f(x) \neq \lim_{x \to a_{0+}}f(x)
    \end{cases}
  \end

Функція в точці x=a має \textbf{усувний розрив першого роду}, за умов що функції рівні але функції в точці a не існує.

\begin{displaymath}
\lim_{x \to a_{0+}}f(x) = \lim_{x \to a_{0-}}f(x) \neq f(a)
\end{displaymath}

Якщо границя зправа або зліва не існує або прямує до нескінченності то у цій точци матимемо \textbf{розрив другого роду}.

\begin{center}\large{\cyr{\textbf{Стрибок функції}}}\end{center}

Стрибком функції в точці розриву $x_0$ є різниця її првосторонньої і лівоторонньої границі.

\begin{displaymath}
\lim_{x \to a_{0+}}f(x) - \lim_{x \to a_{0-}}f(x) \neq f(a)
\end{displaymath}

\begin{enumerate}
\item елементарна функція може мати розрив тільки в окремих точках, але не може бути розривною на певному інтервалі.
\item елементарна функція може мати розрив в точці де вона не визначена за умови, що вона буде визначена хоча би з однієї сторони від цієї точки.
\item неелементарна функція може мати розриви як в точках, де вона невизначена, так і в тих, де вона визначена.
\end{enumerate}

Наприклад, якщо функція задана кількома різними аналітичними виразами (формулами) для різних інтервалів, то на межі стику може бути розривною.
