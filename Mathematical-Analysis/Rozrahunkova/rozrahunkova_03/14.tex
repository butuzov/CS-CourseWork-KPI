Застосовуючи відповідні степеневі ряди, обчисліть з точністю $\varepsilon$ значення функції

$$
  \lg 3, {\qquad}e=10^{-4}
$$

Застосовуючи властивості логарифмів виведемо натуральний логарифм через десятичний

$$
\boxed{ \log_{a} x = \ln{x} \times \log_{a} e }
\Rightarrow
\boxed{ \lg {n} = \log_{10} {n} = \ln{n} \times \log_{10} e }
\Rightarrow
\boxed{ \lg {3} = \ln {3} \times \lg e } {\text{ де }} {\lg e = 0.43429448}
$$

\M{Таким чином ми можемо скористатись рядом для визначення функції $\lg 3$:}

$$
\ln{\Bigg( \dfrac{1+x}{1-x} \Bigg)}
= 2\Bigg( x + \dfrac{x^3}{3} + \dfrac{x^5}{5} + \dfrac{x^7}{7} \ldots + \dfrac{x^{2n-1}}{2n-1} \Bigg)
$$

$$
\dfrac{1+x}{1-x} = 3 \Rightarrow x = \dfrac{1}{2}
$$

\M{тобто:}

$$
\ln{3} = 2 \Bigg( \dfrac{1}{2} + \dfrac{1}{3 \cdot 2^3 } + \dfrac{1}{5 \cdot 2^5} + \dfrac{1}{7 \cdot 2^7 }  + \ldots + \dfrac{1}{(2n-1) \cdot 2^{2n-1}}\Bigg).
$$

\M{Число членів ряду $n$ визначається нерівністю:}

$$
\bigg | r_n\Big( \dfrac{1}{2} \Big) \bigg | \leqslant \dfrac{2}{2^{2n-1} (2n-1)(1-\dfrac{1}{2^2})}  < 10^{-4}
$$

\M{ця нерівність задовольняється при $n=6$. Таким чином:}

$$
\ln{3} \approx 2 \Big(
\dfrac{1}{\cdot 2 } + \dfrac{1}{3 \cdot 2^3 } +
 \dfrac{1}{5 \cdot 2^5 } + \dfrac{1}{7 \cdot 2^7 } +
 \dfrac{1}{9 \cdot 2^9 } + \dfrac{1}{11 \cdot 2^{11} }
\Big) = 1.0985882823773445
$$

\M{І фінально підставивши модуль переходу отримаємо значення функції $\lg{3}$ з точністю $10^{-4}$:}
$$\ln{3} \cdot \lg e = 1.0985882823773445 \cdot 0.43429448 = \boxed{ 0.4771 }$$

\M{\scriptsize{Рахувалось в мові програмування Python}}
