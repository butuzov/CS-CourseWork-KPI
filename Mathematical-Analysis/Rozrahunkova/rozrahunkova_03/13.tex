Розвиньте функцію у ряд Тейлора за степенями $x-a$ та вкажіть область збіжності ряду:

$$
  f(x) = {\dfrac{1}{\sqrt[5]{1+x^2}}} = (1+x^2)^{-\frac{1}{5}}, \qquad a=0
$$

$$
{\text{ Формула ряду Тейлора }}\sum_{n=0}^\infty \dfrac{f^{(n)}(a)}{n!}(x-a)^n
$$

\M{Як ми бачимо ми маємо справу з біномінальним рядом (за умови що ми підставимо $x$ як $x^2$), що у точці $a=0$ перетворєються у ряд Маклорена:}

$$
{\text{ Формула ряду Маклорена }}\sum_{n=0}^\infty \dfrac{f^{(n)}(a)}{n!}x^n
$$

\M{Біномінальний ряд}

$$
(1+x)^{\alpha} = 1+\alpha x + \dfrac{\alpha(\alpha-1)}{2!}x^2 + \dfrac{\alpha(\alpha-1)(\alpha-2)}{3!}x^3 +\ldots + \dfrac{\alpha(\alpha-1)(\alpha-2)}{n!}x^3
$$


$$
(1+x^2)^{-\frac{1}{5}} = 1
  - \dfrac{ \dfrac{x^2}{5} }{ 1!}
  + \dfrac{ \Bigg( -\dfrac{1}{5} \times -\dfrac{6}{5} \Bigg) x^4}{2!}
  + \dfrac{ \Bigg( -\dfrac{1}{5} \times -\dfrac{6}{5} \times -\dfrac{11}{5} \Bigg) x^6} {3!}
  \ldots
  + \dfrac{ \Bigg( \Prod{k=0}{n-1} \big(-\dfrac{1-k}{5}\big) \Bigg) x^{2n} }{n!}
$$

\M{Область збіжності ряду визначаємо із нерівності $|x^2| < 1 \Rightarrow |x| < 1$}

%
% Цм розрахунки була зроблені до того як япроситав пор узагальнення біномінального ряду
%
%
% %
% \M{Знайдемо кілька перших похідних функції:}
% $$
% \begin{array}{lllll}
% f(x)  = & (1+x^2)^{-\frac{1}{5}} & f(0) &=& 1 \\
% \\
% f(x)' = & -\dfrac{2x}{5(1+x^2)^{\frac{6}{5}}} & f(0)' &=& \dfrac{0}{5} =  0 \\
% \\
% f(x)'' = & \dfrac{2(7x^2-5)}{25(1+x^2)^{\frac{11}{5}}} & f(0)'' &=&-\dfrac{10}{25}=-\dfrac{2}{5}\\
% \\
% f(x)''' = & \dfrac{24x(7x^2-15)}{125(1+x^2)^{\frac{16}{5}}} & f(0)''' &=& \dfrac{0}{125}  = 0\\
% \\
% f(x)'''' = & \dfrac{24(119x^4-510x^2+75)}{625(1+x^2)^{\frac{21}{5}}} & f(0)''' &=& \dfrac{1800}{625}  = \dfrac{72}{25} \\
% \\
% f(x)''''' = & \dfrac{528x(119x^4-850x^2+375)}{3125(1+x^2)^{\frac{26}{5}}} & f(0)''''' &=& \dfrac{0}{3125}  = 0\\
% \end{array}
% $$

% похідна третього і більшх порядків була отримана за допомогою сервісу
% wolfram alpha
%
% 1 = D[  -2x/(5(1 + x^2)^(6/5)), x]
% 2 = D[  2(7x^2-5) /( 25(1+x^2)^(11/5) ), x]
% 3 = D[  24x(7x^2-15) /( 125(1+x^2)^(16/5) ), x]
% 4 = D[  24(119x^4-510x^2+75) /( 625(1+x^2)^(21/5) ), x]
% 5 = D[  528x(119x^4-850x^2+375) /( 3125(1+x^2)^(26/5) ), x]

% \M{Оскільки $a=0$, наш ряд Тейлора перетворюється у ряд Маклорена.}
%
% \M{Ряд Маклорена}
%
% $$
%  1 + \dfrac{0}{1!}x - \dfrac{2}{5 \times 2!} x^2 + \dfrac{0}{3!}x^3 + \dfrac{72}{25 \times 4!}x^4 + \dfrac{0}{5!}x^5 \ldots + \dfrac{f^{(n)}(0)}{n!}x^n
%  = 1 + 0 - \dfrac{2}{5}x^2 + 0 + \dfrac{3}{25}x^4 + 0 + \ldots + \dfrac{f^{(n)}(0)}{n!}x^n
% $$
%
% \M{Як бачимо, ми маємо справу з біномінальним рядом:}

% https://www.youtube.com/watch?v=ePx71bN9TqM
% похідна n ного порядку


% Область збіжності https://ru.wikipedia.org/wiki/%D0%A0%D1%8F%D0%B4_%D0%A2%D0%B5%D0%B9%D0%BB%D0%BE%D1%80%D0%B0#%D0%9E%D0%B1%D0%BB%D0%B0%D1%81%D1%82%D1%8C_%D1%81%D1%85%D0%BE%D0%B4%D0%B8%D0%BC%D0%BE%D1%81%D1%82%D0%B8_%D1%80%D1%8F%D0%B4%D0%B0_%D0%A2%D0%B5%D0%B9%D0%BB%D0%BE%D1%80%D0%B0
