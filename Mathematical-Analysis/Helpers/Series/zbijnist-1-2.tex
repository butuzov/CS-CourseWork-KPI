\M{\large{\textbf{Знако\_змінність та Ознаки збіжності таких рядів}}}
\begin{multicols}{2}
  \M{\textbf{ЗнакоПостійні}}
  $$
  \boxed{ \sum_{n=1}^\infty a_n = a_1 + a_2 + a_3 + a_4 \ldots + a_n }
  $$

  \begin{enumerate}
    \item 1ша ознака збіжності
    \item 2га ознака збіжності
    \item Ознака КОШІ
    \item Ознака ДАЛАМБЕРА
    \item Інтегральна ознака
    \end{enumerate}

  \columnbreak

  \M{\textbf{ЗнакоЗмінні}}
  $$
  \boxed{ \sum_{n=1}^\infty -1^{n+1}a_n = a_1 - a_2 + a_3 - a_4 \ldots + (-1)^{n+1}a_n }
  $$

  \begin{enumerate}
    \item Ознака Лейбніца
  \end{enumerate}
\end{multicols}

\begin{multicols}{2}

  \M{\large{\textbf{1ша ознака збіжності \\ Ознака Порівняння}}}
  \begin{center}
    Порівняємо членів рядів $\Sum{n=1}{\infty} a$ та $\Sum{n=1}{\infty} b$

    За умови що $ \boxed{ \forall a_n \leqslant b_n } $ (кожен член ряду $a$ менше або дорівнює кожному члену ряду $b$), тоді якщо збігається $\Sum{n=1}{\infty} b$ буде збігається і $\Sum{n=1}{\infty} a$, і навпаки.
  \end{center}

  \columnbreak

  \M{\large{\textbf{2га ознака збіжності \\ Умова Граничної Ознаки}}}

  \begin{center}
  Якщо існує границя $k = \boxed{\Lim{n \to \infty} \dfrac{a_n}{b_n}}$ кінцева і відмінна від нуля границя , то обидва ряди $\Sum{n=1}{\infty} a$ b $\Sum{n=1}{\infty} b$ одночасно сходяться або розходяться. Що може служити достатньою ознакою збіжності (якщо порівнюбвати з вже відомим рядом)
  \end{center}

\end{multicols}
