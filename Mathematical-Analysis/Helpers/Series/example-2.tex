$$
2) \sum_{n=1}^{\infty} \dfrac{1}{3n+3}
  \qquad
  \Bigg|
    \begin{array}{rlrl}
      a_n = & \dfrac{1}{3n+3} & b_n = & \dfrac{1}{n} \\
      \end{array}
      \Bigg| \Rightarrow
    \lim_{n \to \infty} \dfrac{a_n}{b_n} = \lim_{n \to \infty} \dfrac{n}{3n+3} = \dfrac{1}{3} \lim_{n \to \infty} \dfrac{1}{1+1/n} = \dfrac{1}{3}
$$

\M{Ряд є \textbf{розбіжним за другою ознакою збіжності} - оскільки границею від частки загального елементу нашого ряду на загальний елемент гармонічного ряду є кінцеве і відмінне від нуля число, а значить наш ряд має туж характеристику що і ряд щагальний елемент якого ми вкористали для діленя (тобто "розбіжний").}
