\documentclass[12pt,a4paper,leqno]{article}
\usepackage[latin1]{inputenc}
\usepackage[left=1in,right=1in,top=1in,bottom=1in]{geometry}
\usepackage{amsmath}
\usepackage{amsfonts}
\usepackage{amssymb}
\usepackage{url}
\usepackage{hyperref}
\usepackage[hang,flushmargin]{footmisc}

\date{\today} % delete this line to display the current date


\hypersetup{
    pdftitle={Integral Table from http://integral-table.com},    % title
    pdfauthor={Shapiro},     % author
    pdfsubject={Table of Integrals},   % subject of the document
    pdfcreator={pdftex},   % creator of the document
    pdfproducer={Texmaker}, % producer of the document
    pdfkeywords={CSUN, Integrals, Table of Integrals, Math 280, Math 351, Differential Equations}, % list of keywords
    colorlinks=true,       % false: boxed links; true: colored links
    linkcolor=red,          % color of internal links
    citecolor=red,        % color of links to bibliography
    filecolor=red,      % color of file links
    urlcolor=red           % color of external links
}

\newcommand{\dx}{\hspace{2pt}dx}
\newcommand{\dd}[1]{\hspace{2pt}d#1}
\usepackage{multicol}

\usepackage{hyperref}
\hypersetup{
    pdftitle={Integral Table from http://integral-table.com},    % title
    pdfauthor={Shapiro},     % author
    pdfsubject={Table of Integrals},   % subject of the document
    pdfcreator={pdftex},   % creator of the document
    pdfproducer={Texmaker}, % producer of the document
    pdfkeywords={CSUN, Integrals, Table of Integrals, Math 280, Math 351, Differential Equations}, % list of keywords
    colorlinks=true,       % false: boxed links; true: colored links
    linkcolor=red,          % color of internal links
    citecolor=red,        % color of links to bibliography
    filecolor=red,      % color of file links
    urlcolor=red           % color of external links
}

\begin{document}
  \pagestyle{empty}


  \section*{Basic Forms}

    \begin{equation}
    \int x^n dx = \frac{1}{n+1}x^{n+1},\hspace{1ex}n\neq -1
    \end{equation}

    \begin{equation}
    \int \frac{1}{x}dx = \ln |x|
    \end{equation}

    \begin{equation}
    \int u dv = uv - \int v du
    \end{equation}

    \begin{equation}
    \int \frac{1}{ax+b}dx = \frac{1}{a} \ln |ax + b|
    \end{equation}

    \section* {Integrals of Rational Functions}

    \begin{equation}
    \int \frac{1}{(x+a)^2}dx = -\frac{1}{x+a}
    \end{equation}

    \begin{equation}
    \int (x+a)^n dx = \frac{(x+a)^{n+1}}{n+1}, n\ne -1
    \end{equation}

    \begin{equation}
    \int x(x+a)^n dx = \frac{(x+a)^{n+1} ( (n+1)x-a)}{(n+1)(n+2)}
    \end{equation}

    \begin{equation}
    \int \frac{1}{1+x^2}dx = \tan^{-1}x
    \end{equation}


    \begin{equation}
    \int \frac{1}{a^2+x^2}dx = \frac{1}{a}\tan^{-1}\frac{x}{a}
    \end{equation}

    \begin{equation}
    \int \frac{x}{a^2+x^2}dx = \frac{1}{2}\ln|a^2+x^2|
    \end{equation}

    \begin{equation}
    \int \frac{x^2}{a^2+x^2}dx = x-a\tan^{-1}\frac{x}{a}
    \end{equation}

    \begin{equation}
    \int \frac{x^3}{a^2+x^2}dx = \frac{1}{2}x^2-\frac{1}{2}a^2\ln|a^2+x^2|
    \end{equation}

    \begin{equation}
    \int \frac{1}{ax^2+bx+c}dx = \frac{2}{\sqrt{4ac-b^2}}\tan^{-1}\frac{2ax+b}{\sqrt{4ac-b^2}}
    \end{equation}

    \begin{equation}
    \int \frac{1}{(x+a)(x+b)}dx = \frac{1}{b-a}\ln\frac{a+x}{b+x}, \text{ } a\ne b
    \end{equation}

    \begin{equation}
    \int \frac{x}{(x+a)^2}dx = \frac{a}{a+x}+\ln |a+x|
    \end{equation}


    \begin{equation}
    \int \frac{x}{ax^2+bx+c}dx = \frac{1}{2a}\ln|ax^2+bx+c|
    -\frac{b}{a\sqrt{4ac-b^2}}\tan^{-1}\frac{2ax+b}{\sqrt{4ac-b^2}}
    \end{equation}

    \section*{Integrals with Roots}


    \begin{equation}
    \int \sqrt{x-a}\ dx = \frac{2}{3}(x-a)^{3/2}
    \end{equation}



    \begin{equation}
    \int \frac{1}{\sqrt{x\pm a}}\ dx = 2\sqrt{x\pm a}
    \end{equation}

    \begin{equation}\label{eq:Rigo}
    \int \frac{1}{\sqrt{a-x}}\ dx = -2\sqrt{a-x}
    \end{equation}


    \begin{equation}\label{eq:Gilmore}
    \int x\sqrt{x-a}\ dx =
    \left\{
    \begin{array}{l}
      \dfrac{2 a}{3} \left({x-a}\right)^{3/2} +\dfrac{2 }{5}\left( {x-a}\right)^{5/2},\text{ or} \\
      \dfrac{2}{3} x(x-a)^{3/2} - \frac{4}{15} (x-a)^{5/2}, \text{ or} \\
      \dfrac{2}{15}(2a+3x)(x-a)^{3/2}
    \end{array}
    \right.
    \end{equation}

    \begin{equation}
    \int \sqrt{ax+b}\ dx = \left(\frac{2b}{3a}+\frac{2x}{3}\right)\sqrt{ax+b}
    \end{equation}

    \begin{equation}
    \int (ax+b)^{3/2}\ dx =\frac{2}{5a}(ax+b)^{5/2}
    \end{equation}

    \begin{equation}\label{eq:Weems}
    \int \frac{x}{\sqrt{x\pm a} } \ dx = \frac{2}{3}(x\mp 2a)\sqrt{x\pm a}
    \end{equation}

    \begin{equation}
    \int \sqrt{\frac{x}{a-x}}\ dx =  -\sqrt{x(a-x)}
    -a\tan^{-1}\frac{\sqrt{x(a-x)}}{x-a}
    \end{equation}

    \begin{equation}
    \int \sqrt{\frac{x}{a+x}}\ dx =  \sqrt{x(a+x)}
    -a\ln \left [ \sqrt{x} + \sqrt{x+a}\right]
    \end{equation}

    \begin{equation}
    \int x \sqrt{ax + b}\ dx =
    \frac{2}{15 a^2}(-2b^2+abx + 3 a^2 x^2)
    \sqrt{ax+b}
    \end{equation}

    \begin{equation}
    \int \sqrt{x(ax+b)}\ dx = \frac{1}{4a^{3/2}}\left[(2ax + b)\sqrt{ax(ax+b)}
    -b^2 \ln \left| a\sqrt{x} + \sqrt{a(ax+b)} \right| \right ]
    \end{equation}

    \begin{equation}
    \int \sqrt{x^3(ax+b)} \ dx =\left [
    \frac{b}{12a}-
    \frac{b^2}{8a^2x}+
    \frac{x}{3}\right]
    \sqrt{x^3(ax+b)}  +
    \frac{b^3}{8a^{5/2}}\ln \left | a\sqrt{x} + \sqrt{a(ax+b)} \right |
    \end{equation}

    \begin{equation}
    \int\sqrt{x^2 \pm a^2}\ dx = \frac{1}{2}x\sqrt{x^2\pm a^2}
    \pm\frac{1}{2}a^2 \ln \left | x + \sqrt{x^2\pm a^2} \right |
    \end{equation}



   \begin{equation}
   \int  \sqrt{a^2 - x^2}\ dx = \frac{1}{2} x \sqrt{a^2-x^2}
   +\frac{1}{2}a^2\tan^{-1}\frac{x}{\sqrt{a^2-x^2}}
   \end{equation}

   \begin{equation}
   \int  x \sqrt{x^2 \pm a^2}\ dx= \frac{1}{3}\left ( x^2 \pm a^2 \right)^{3/2}
   \end{equation}

   \begin{equation}
   \int \frac{1}{\sqrt{x^2 \pm a^2}}\ dx = \ln \left | x + \sqrt{x^2 \pm a^2} \right |
   \end{equation}

   \begin{equation}
   \int \frac{1}{\sqrt{a^2 - x^2}}\ dx = \sin^{-1}\frac{x}{a}
   \end{equation}

   \begin{equation}
   \int \frac{x}{\sqrt{x^2\pm a^2}}\ dx = \sqrt{x^2 \pm a^2}
   \end{equation}

   \begin{equation}
   \int \frac{x}{\sqrt{a^2-x^2}}\ dx = -\sqrt{a^2-x^2}
   \end{equation}

   \begin{equation}\label{eq:Russ}
   \int \frac{x^2}{\sqrt{x^2 \pm a^2}}\ dx = \frac{1}{2}x\sqrt{x^2 \pm a^2}
   \mp \frac{1}{2}a^2 \ln \left| x + \sqrt{x^2\pm a^2} \right |
   \end{equation}

   \begin{equation}\label{eq:Winokur1}
   \int \sqrt{a x^2 + b x + c}\ dx =
   \frac{b+2ax}{4a}\sqrt{ax^2+bx+c}
   +
   \frac{4ac-b^2}{8a^{3/2}}\ln \left| 2ax + b + 2\sqrt{a(ax^2+bx^+c)}\right |
   \end{equation}

   \begin{equation}\label{eq:Larry-Morris}\begin{split}
   \int &x \sqrt{a x^2 + bx + c}\ dx = \frac{1}{48a^{5/2}}\left (
   2 \sqrt{a} \sqrt{ax^2+bx+c}
   \right .
     \left( - 3b^2 + 2 abx + 8 a(c+ax^2) \right)
   \\ &  \left.
    + 3(b^3-4abc)\ln \left|b + 2ax + 2\sqrt{a}\sqrt{ax^2+bx+c} \right| \right)
    \end{split}
   \end{equation}

   \begin{equation}
   \int\frac{1}{\sqrt{ax^2+bx+c}}\ dx=
   \frac{1}{\sqrt{a}}\ln \left| 2ax+b + 2 \sqrt{a(ax^2+bx+c)} \right |
   \end{equation}

   \begin{equation}\label{eq:Duley}
   \int \frac{x}{\sqrt{ax^2+bx+c}}\ dx=
   \frac{1}{a}\sqrt{ax^2+bx + c}
   -
   \frac{b}{2a^{3/2}}\ln \left| 2ax+b + 2 \sqrt{a(ax^2+bx+c)} \right |
   \end{equation}

   \begin{equation}\label{eq:Winokur2}
   \int\frac{dx}{(a^2+x^2)^{3/2}}=\frac{x}{a^2\sqrt{a^2+x^2}}
   \end{equation}



   \section*{Integrals with Logarithms}

   \begin{equation}
   \int \ln ax\  dx = x \ln ax - x
   \end{equation}

   \begin{equation}
   \int x \ln x \ dx = \frac{1}{2} x^2 \ln x-\frac{x^2}{4}
   \end{equation}

   \begin{equation}
   \int x^2 \ln x \ dx = \frac{1}{3} x^3 \ln x-\frac{x^3}{9}
   \end{equation}

   \begin{equation}
   \int x^n \ln x\ dx = x^{n+1}\left( \dfrac{\ln x}{n+1}-\dfrac{1}{(n+1)^2}\right),\hspace{2ex} n\neq -1
   \end{equation}


   \begin{equation}
   \int \frac{\ln ax}{x}\ dx = \frac{1}{2}\left ( \ln ax \right)^2
   \end{equation}

   \begin{equation}
   \int \frac{\ln x}{x^2}\ dx = -\frac{1}{x}-\frac{\ln x}{x}
   \end{equation}

   \begin{equation}
   \int \ln (ax + b) \ dx = \left ( x + \frac{b}{a} \right) \ln (ax+b) - x , a\ne 0
   \end{equation}

   \begin{equation}
   \int \ln  ( x^2 + a^2 )\hspace{.5ex} {dx} = x \ln (x^2 + a^2  ) +2a\tan^{-1} \frac{x}{a} - 2x
   \end{equation}

   \begin{equation}
   \int \ln  ( x^2 - a^2 )\hspace{.5ex} {dx} = x \ln (x^2 - a^2  ) +a\ln \frac{x+a}{x-a} - 2x
   \end{equation}

   \begin{equation}
   \int \ln \left ( ax^2 + bx + c\right) \ dx  = \frac{1}{a}\sqrt{4ac-b^2}\tan^{-1}\frac{2ax+b}{\sqrt{4ac-b^2}}
   -2x + \left( \frac{b}{2a}+x \right )\ln \left (ax^2+bx+c \right)
   \end{equation}

   \begin{equation}
   \int x \ln (ax + b)\ dx = \frac{bx}{2a}-\frac{1}{4}x^2
   +\frac{1}{2}\left(x^2-\frac{b^2}{a^2}\right)\ln (ax+b)
   \end{equation}

   \begin{equation}
   \int x \ln \left ( a^2 - b^2 x^2 \right )\ dx = -\frac{1}{2}x^2+
   \frac{1}{2}\left( x^2 - \frac{a^2}{b^2} \right ) \ln \left (a^2 -b^2 x^2 \right)
   \end{equation}

   \begin{equation}
   \int (\ln x)^2\ dx = 2x - 2x \ln x + x (\ln x)^2
   \end{equation}


   \begin{equation}
   \int (\ln x)^3\ dx = -6 x+x (\ln x)^3-3 x (\ln x)^2+6 x \ln x
   \end{equation}


   \begin{equation}
   \int x (\ln x)^2\ dx = \frac{x^2}{4}+\frac{1}{2} x^2 (\ln x)^2-\frac{1}{2} x^2 \ln x
   \end{equation}

   \begin{equation}
   \int x^2 (\ln x)^2\ dx = \frac{2 x^3}{27}+\frac{1}{3} x^3 (\ln x)^2-\frac{2}{9} x^3 \ln x
   \end{equation}


    \section*{Integrals with Exponentials}

   \begin{equation}
   \int e^{ax}\ dx = \frac{1}{a}e^{ax}
   \end{equation}

   \begin{equation}\label{eq:ajoy}
   \int \sqrt{x} e^{ax}\ dx = \frac{1}{a}\sqrt{x}e^{ax}
   +\frac{i\sqrt{\pi}}{2a^{3/2}}
   \text{erf}\left(i\sqrt{ax}\right),
   \text{ where erf}(x)=\frac{2}{\sqrt{\pi}}\int_0^x e^{-t^2}dt
   \end{equation}

   \begin{equation}
   \int x e^x\ dx = (x-1) e^x
   \end{equation}

   \begin{equation}
   \int x e^{ax}\ dx = \left(\frac{x}{a}-\frac{1}{a^2}\right) e^{ax}
   \end{equation}

   \begin{equation}
   \int x^2 e^{x}\ dx = \left(x^2 - 2x + 2\right) e^{x}
   \end{equation}

   \begin{equation}
   \int x^2 e^{ax}\ dx = \left(\frac{x^2}{a}-\frac{2x}{a^2}+\frac{2}{a^3}\right) e^{ax}
   \end{equation}

   \begin{equation}
   \int x^3 e^{x}\ dx = \left(x^3-3x^2 + 6x - 6\right) e^{x}
   \end{equation}

   \begin{equation}\label{eq:swift1}
   \int x^n e^{ax}\ dx = \dfrac{x^n e^{ax}}{a} -
   \dfrac{n}{a}\int x^{n-1}e^{ax}\hspace{1pt}\text{d}x
   \end{equation}

   \begin{equation}\label{eq:ebke}
   \int x^n e^{ax}\ dx = \frac{(-1)^n}{a^{n+1}}\Gamma[1+n,-ax],
    \text{ where } \Gamma(a,x)=\int_x^{\infty} t^{a-1}e^{-t}\hspace{2pt}\text{d}t
    \end{equation}

   \begin{equation}\label{eq:swift2}
   \int e^{ax^2}\ dx = -\frac{i\sqrt{\pi}}{2\sqrt{a}}\text{erf}\left(ix\sqrt{a}\right)
   \end{equation}

   \begin{equation}\label{eq:swift3}
   \int e^{-ax^2}\ dx = \frac{\sqrt{\pi}}{2\sqrt{a}}\text{erf}\left(x\sqrt{a}\right)
   \end{equation}

   \begin{equation}\label{eq:qarles1}
   \int x e^{-ax^2}\ {dx} = -\dfrac{1}{2a}e^{-ax^2}
   \end{equation}

   \begin{equation}\label{eq:qarles2}
   \int x^2 e^{-ax^2}\ {dx} = \dfrac{1}{4}\sqrt{\dfrac{\pi}{a^3}}\text{erf}(x\sqrt{a}) -\dfrac{x}{2a}e^{-ax^2}
   \end{equation}


   \section* {Integrals with Trigonometric Functions}

   \begin{equation}
   \int \sin ax \ dx = -\frac{1}{a} \cos ax
   \end{equation}

   \begin{equation}
   \int \sin^2 ax\  dx = \frac{x}{2} - \frac{\sin 2ax} {4a}
   \end{equation}

   \begin{equation}
   \int \sin^3 ax \ dx = -\frac{3 \cos ax}{4a} + \frac{\cos 3ax} {12a}
   \end{equation}

   \begin{equation}
   \int \sin^n ax \ dx =
    -\frac{1}{a}{\cos ax} \hspace{2mm}{_2F_1}\left[
   \frac{1}{2}, \frac{1-n}{2}, \frac{3}{2}, \cos^2 ax
   \right]
   \end{equation}



   \begin{equation}
   \int \cos ax\ dx= \frac{1}{a} \sin ax
   \end{equation}

   \begin{equation}
   \int \cos^2 ax\ dx = \frac{x}{2}+\frac{ \sin 2ax}{4a}
   \end{equation}

   \begin{equation}
   \int \cos^3 ax dx = \frac{3 \sin ax}{4a}+\frac{ \sin 3ax}{12a}
   \end{equation}

   \begin{equation}
   \int \cos^p ax dx  = -\frac{1}{a(1+p)}{\cos^{1+p} ax} \times
   {_2F_1}\left[
   \frac{1+p}{2}, \frac{1}{2}, \frac{3+p}{2}, \cos^2 ax
   \right]
   \end{equation}

   \begin{equation}\label{eq:veky}
   \int \cos x \sin x\ dx = \frac{1}{2}\sin^2 x + c_1 = -\frac{1}{2} \cos^2x + c_2 = -\frac{1}{4} \cos 2x + c_3
   \end{equation}

   \begin{equation}
   \int \cos ax \sin bx\ dx = \frac{\cos[(a-b) x]}{2(a-b)} -
    \frac{\cos[(a+b)x]}{2(a+b)} , a\ne b
   \end{equation}

   \begin{equation}
   \int \sin^2 ax \cos bx\ dx =
   -\frac{\sin[(2a-b)x]}{4(2a-b)}
   + \frac{\sin bx}{2b}
   - \frac{\sin[(2a+b)x]}{4(2a+b)}
   \end{equation}

   \begin{equation}
   \int \sin^2 x \cos x\ dx = \frac{1}{3} \sin^3 x
   \end{equation}

   \begin{equation}
   \int \cos^2 ax \sin bx\ dx = \frac{\cos[(2a-b)x]}{4(2a-b)}
   - \frac{\cos bx}{2b}
    - \frac{\cos[(2a+b)x]}{4(2a+b)}
   \end{equation}

   \begin{equation}
   \int \cos^2 ax \sin ax\ dx = -\frac{1}{3a}\cos^3{ax}
   \end{equation}



   \begin{equation}
   \int \sin^2 ax \cos^2 bx dx = \frac{x}{4}
   -\frac{\sin 2ax}{8a}-
   \frac{\sin[2(a-b)x]}{16(a-b)}
   +\frac{\sin 2bx}{8b}-
   \frac{\sin[2(a+b)x]}{16(a+b)}
   \end{equation}

   \begin{equation}
   \int \sin^2 ax \cos^2 ax\ dx = \frac{x}{8}-\frac{\sin 4ax}{32a}
   \end{equation}

   \begin{equation}
   \int \tan ax\ dx = -\frac{1}{a} \ln \cos ax
   \end{equation}

   \begin{equation}
   \int \tan^2 ax\ dx = -x + \frac{1}{a} \tan ax
   \end{equation}

   \begin{equation}
   \int \tan^n ax\ dx =
   \frac{\tan^{n+1} ax }{a(1+n)} \times
    {_2}F_1\left( \frac{n+1}{2},
   1, \frac{n+3}{2}, -\tan^2 ax \right)
   \end{equation}

   \begin{equation}
   \int \tan^3 ax dx = \frac{1}{a} \ln \cos ax + \frac{1}{2a}\sec^2 ax
   \end{equation}

   \begin{equation}
   \int \sec x \ dx = \ln | \sec x + \tan x | = 2 \tanh^{-1} \left (\tan \frac{x}{2} \right)
   \end{equation}

   \begin{equation}
   \int \sec^2 ax\ dx = \frac{1}{a} \tan ax
   \end{equation}

   \begin{equation}\label{eq:Kloeppel}
   \int \sec^3 x \ {dx} = \frac{1}{2} \sec x \tan x + \frac{1}{2}\ln | \sec x + \tan x |
   \end{equation}

   \begin{equation}
   \int \sec x \tan x\ dx = \sec x
   \end{equation}

   \begin{equation}
   \int \sec^2 x \tan x\ dx = \frac{1}{2} \sec^2 x
   \end{equation}

   \begin{equation}
   \int \sec^n x \tan x \ dx = \frac{1}{n} \sec^n x , n\ne 0
   \end{equation}

   \begin{equation}
   \int \csc x\ dx = \ln \left | \tan \frac{x}{2} \right|  = \ln | \csc x - \cot x| + C
   \end{equation}

   \begin{equation}
   \int \csc^2 ax\ dx = -\frac{1}{a} \cot ax
   \end{equation}

   \begin{equation}
   \int \csc^3 x\ dx = -\frac{1}{2}\cot x \csc x + \frac{1}{2} \ln | \csc x - \cot x |
   \end{equation}

   \begin{equation}
   \int \csc^nx \cot x\ dx = -\frac{1}{n}\csc^n x, n\ne 0
   \end{equation}

   \begin{equation}
   \int \sec x \csc x \ dx = \ln | \tan x |
   \end{equation}


   \section*{Products of Trigonometric Functions and Monomials}


  \begin{equation}
  \int x \cos x \ dx = \cos x + x \sin x
  \end{equation}

  \begin{equation}
  \int x \cos ax \ dx = \frac{1}{a^2} \cos ax + \frac{x}{a} \sin ax
  \end{equation}

  \begin{equation}
  \int x^2 \cos x \ dx = 2 x \cos x + \left ( x^2 - 2 \right ) \sin x
  \end{equation}

  \begin{equation}
  \int x^2 \cos ax \ dx = \frac{2 x \cos ax }{a^2} + \frac{ a^2 x^2 - 2  }{a^3} \sin ax
  \end{equation}

  \begin{equation}
  \int  x^n \cos x dx =
  -\frac{1}{2}(i)^{n+1}\left [ \Gamma(n+1, -ix)
  % \right . \nonumber \\ & \left .
  + (-1)^n \Gamma(n+1, ix)\right]
  \end{equation}

  \begin{equation}
  \int x^n \cos ax \ dx =
   \frac{1}{2}(ia)^{1-n}\left [ (-1)^n  \Gamma(n+1, -iax)
  % \right. \nonumber \\ & \left.
   -\Gamma(n+1, ixa)\right]
  \end{equation}

  \begin{equation}
  \int x \sin x\ dx = -x \cos x + \sin x
  \end{equation}

  \begin{equation}
  \int x \sin ax\ dx = -\frac{x \cos ax}{a} + \frac{\sin ax}{a^2}
  \end{equation}

  \begin{equation}
  \int x^2 \sin x\ dx = \left(2-x^2\right) \cos x + 2 x \sin x
  \end{equation}

  \begin{equation}
  \int x^2 \sin ax\ dx =\frac{2-a^2x^2}{a^3}\cos ax +\frac{ 2 x \sin ax}{a^2}
  \end{equation}

  \begin{equation}\label{eq:xul}
  \int x^n \sin x \ dx = -\frac{1}{2}(i)^n\left[ \Gamma(n+1, -ix)
  %\right. \nonumber \\ & \left.
   - (-1)^n\Gamma(n+1, -ix)\right]
  \end{equation}

  \begin{equation}
  \int x \cos^2 x \ dx = \frac{x^2}{4}+\frac{1}{8}\cos 2x + \frac{1}{4} x \sin 2x
  \end{equation}

  \begin{equation}
  \int x \sin^2 x \ dx = \frac{x^2}{4}-\frac{1}{8}\cos 2x - \frac{1}{4} x \sin 2x
  \end{equation}

  \begin{equation}
  \int x \tan^2 x \ dx = -\frac{x^2}{2} + \ln \cos x + x \tan x
  \end{equation}

  \begin{equation}
  \int x \sec^2 x \ dx = \ln \cos x + x \tan x
  \end{equation}

  \section*{Products of Trigonometric Functions and Exponentials}

  \begin{equation}
  \int e^x \sin x \ dx = \frac{1}{2}e^x (\sin x - \cos x)
  \end{equation}

  \begin{equation}\label{eq:ritzert}
  \int e^{bx} \sin ax\ dx = \frac{1}{a^2+b^2}e^{bx} (b\sin ax - a\cos ax)
  \end{equation}

  \begin{equation}
  \int e^x \cos x\ dx = \frac{1}{2}e^x (\sin x + \cos x)
  \end{equation}

  \begin{equation}
  \int e^{bx} \cos ax\ dx = \frac{1}{a^2 + b^2} e^{bx} ( a \sin ax + b \cos ax )
  \end{equation}

  \begin{equation}
  \int x e^x \sin x\ dx = \frac{1}{2}e^x (\cos x - x \cos x + x \sin x)
  \end{equation}

  \begin{equation}
  \int x e^x \cos x\ dx = \frac{1}{2}e^x (x \cos x
  - \sin x + x \sin x)
  \end{equation}

   \section*{Integrals of Hyperbolic Functions}

  \begin{equation}
  \int \cosh ax\ dx =\frac{1}{a} \sinh ax
  \end{equation}

  \begin{equation}
  \int e^{ax}  \cosh bx \ dx =
  \begin{cases}
  \displaystyle{\frac{e^{ax}}{a^2-b^2} }[ a \cosh bx - b \sinh bx ]  & a\ne b \\
  \displaystyle{\frac{e^{2ax}}{4a} + \frac{x}{2}}  & a = b
  \end{cases}
  \end{equation}

  \begin{equation}
  \int \sinh ax\ dx = \frac{1}{a} \cosh ax
  \end{equation}

  \begin{equation}
  \int e^{ax} \sinh bx \ dx =
  \begin{cases}
  \displaystyle{\frac{e^{ax}}{a^2-b^2} }[ -b \cosh bx + a \sinh bx ]  & a\ne b \\
  \displaystyle{\frac{e^{2ax}}{4a} - \frac{x}{2}}  & a = b
  \end{cases}
  \end{equation}


  \begin{equation}\label{eq:yates}
  \int  \tanh ax\hspace{1.5pt} dx =\frac{1}{a} \ln \cosh ax
  \end{equation}

  \begin{equation}\label{eq:dewitt}
  \int  e^{ax} \tanh bx\ dx =
  \begin{cases}
  \displaystyle{ \frac{ e^{(a+2b)x}}{(a+2b)}
  {_2F_1}\left[ 1+\frac{a}{2b},1,2+\frac{a}{2b}, -e^{2bx}\right] }& \\
  \displaystyle{
  \hspace{1cm}-\frac{1}{a}e^{ax}{_2F_1}\left[ 1, \frac{a}{2b},1+\frac{a}{2b}, -e^{2bx}\right]
  }
   & a\ne b \\
  \displaystyle{\frac{e^{ax}-2\tan^{-1}[e^{ax}]}{a} } & a = b
  \end{cases}
  \end{equation}



  \begin{equation}
  \int \cos ax \cosh bx\ dx =
  \frac{1}{a^2 + b^2} \left[
  a \sin ax \cosh bx  + b \cos ax \sinh bx
  \right]
  \end{equation}

  \begin{equation}
  \int \cos ax \sinh bx\ dx =
  \frac{1}{a^2 + b^2} \left[
  b \cos ax \cosh bx +
   a \sin ax \sinh bx
  \right]
  \end{equation}

  \begin{equation}
  \int \sin ax \cosh bx \ dx =
  \frac{1}{a^2 + b^2} \left[
  -a \cos ax \cosh bx +
   b \sin ax \sinh bx
  \right]
  \end{equation}

  \begin{equation}
  \int \sin ax \sinh bx \ dx =
  \frac{1}{a^2 + b^2} \left[
  b \cosh bx \sin ax -
   a \cos ax \sinh bx
  \right]
  \end{equation}

  \begin{equation}
  \int \sinh ax \cosh ax dx=
  \frac{1}{4a}\left[
  -2ax + \sinh 2ax \right]
  \end{equation}

  \begin{equation}
  \int \sinh ax \cosh bx \ dx =
  \frac{1}{b^2-a^2}\left[
  b \cosh bx \sinh ax
  - a \cosh ax \sinh bx \right]
  \end{equation}


  \begin{footnotesize}
    \copyright \ 2014. From  \url{http://integral-table.com}, last revised \today. This material
    is provided as is without warranty or representation about the accuracy, correctness or suitability of this material for any purpose. This work is licensed under the Creative Commons Attribution-Noncommercial-Share Alike 3.0 United States License. To view a copy of this license, visit \url{http://creativecommons.org/licenses/by-nc-sa/3.0/} or send a letter to Creative Commons, 171 Second Street, Suite 300, San Francisco, California, 94105, USA.
  \end{footnotesize}

\end{document}
