\begin{center}\Large{\cyr{\textbf{Границі функції}}}\end{center}

Стале число $a$ називають границею послідовності $a=a_n$ якщо для кожного додатнього числа $\epsilon$ яким би малим воно не було існує $N$ що всі значення $a_n$, в яких $n\geN$ виконується \mbox{ $|a_n - a| < \epsilon$}

\begin{displaymath}
  \lim_{n \to \infty} a_n  = a
\end{displaymath}

Номер $N$ залежить від вибору числа $\epsilon$. При зменшенні $\epsilon$ число  $N$ буде збільшуватись. Тобто, чим більш близьких значень $a_n$ до $a$ вимагати, тим ймовірніше більш далеких значення ряду доведеться розглядати.

\begin{teorem}
\noindent\textbf{Визначення}: Послідовніть $a_n$ називається нескінченно малою якщо гранця цієї послідовності прямує до 0, і навпаки нескінченно великою якщо границя праямує до $\infty$.
\end{teorem}

\begin{teorem}
\noindent\textbf{Теорема}: Якщо послідовність $a_n$ нескінченно мала, то $\frac{1}{a_n}$ нескінченно велика, і навпаки.
\end{teorem}
