\begin{center}
  \Large{\cyr{\textbf{Завдання №2}}}
\end{center}


\begin{quote}
  \textbf{Завдання}: Проілюструйте процес упорядкування послідовності чисел \\ ($\footnotesize\texttt{[20, 4, 45, 1, 2, 6, 88]}$) алгоритмом сортування вставкою:
\end{quote}

\begin{enumerate}
  \item Перевірка довжини масиву
  \item Якщо довжина менше або дорівнює 1 - преривання функції сортування твердженням "return"
  \item Початок ітерації елемента масиву за індексом ``1'' ( $\normalsize\texttt{array[1]}$ ) що має значення ``4''
  \item Значення ``20'' елемента масиву за індексом 0 ($\normalsize\texttt{array[0]}$) скопійовано в елемент масиву за індексом 1 ($\normalsize\texttt{array[1]}$)
  \item Eлемента масиву за індексом 0 ($\normalsize\texttt{array[0]}$) набув нового значення ``4'' (найменше число данної ітерації)
  \item Початок ітерації елемента масиву за індексом ``2'' ($\normalsize\texttt{array[2]}$) що має значення ``45''
  \item Eлемента масиву за індексом 2 ($\normalsize\texttt{array[2]}$) не потребує нового значення оскільики є найбішим значенням сортування цієї ітерації.
  \item Початок ітерації елемента масиву за індексом ``3'' ($\normalsize\texttt{array[3]}$) що має значення ``1''
  \item Значення ``45'' елемента масиву за індексом 2 ($\normalsize\texttt{array[2]}$) скопійовано в елемент масиву за індексом 3 ($\normalsize\texttt{array[3]}$)
  \item Значення ``20'' елемента масиву за індексом 1 ($\normalsize\texttt{array[1]}$) скопійовано в елемент масиву за індексом 2 ($\normalsize\texttt{array[2]}$)
  \item Значення ``4'' елемента масиву за індексом 0 ($\normalsize\texttt{array[0]}$) скопійовано в елемент масиву за індексом 1 ($\normalsize\texttt{array[1]}$)
  \item Eлемента масиву за індексом 0 ($\normalsize\texttt{array[0]}$) набув нового значення ``1'' (найменше число данної ітерації)
  \item Початок ітерації елемента масиву за індексом ``4'' ($\normalsize\texttt{array[4]}$) що має значення ``2''
  \item Значення ``45'' елемента масиву за індексом 3 ($\normalsize\texttt{array[3]}$) скопійовано в елемент масиву за індексом 4 ($\normalsize\texttt{array[4]}$)
  \item Значення ``20'' елемента масиву за індексом 2 ($\normalsize\texttt{array[2]}$) скопійовано в елемент масиву за індексом 3 ($\normalsize\texttt{array[3]}$)
  \item Значення ``4'' елемента масиву за індексом 1 ($\normalsize\texttt{array[1]}$) скопійовано в елемент масиву за індексом 2 ($\normalsize\texttt{array[2]}$)
  \item Eлемента масиву за індексом 1 ($\normalsize\texttt{array[1]}$) набув нового значення ``2'' (найменше число данної ітерації)
  \item Початок ітерації елемента масиву за індексом ``5'' ($\normalsize\texttt{array[5]}$) що має значення ``6''
  \item Значення ``45'' елемента масиву за індексом 4 ($\normalsize\texttt{array[4]}$) скопійовано в елемент масиву за індексом 5 ($\normalsize\texttt{array[5]}$)
  \item Значення ``20'' елемента масиву за індексом 3 ($\normalsize\texttt{array[3]}$) скопійовано в елемент масиву за індексом 4 ($\normalsize\texttt{array[4]}$)
  \item Eлемента масиву за індексом 3 ($\normalsize\texttt{array[3]}$) набув нового значення ``6'' (найменше число данної ітерації)
  \item Початок ітерації елемента масиву за індексом ``6'' ($\normalsize\texttt{array[6]}$) що має значення ``88''
  \item Eлемента масиву за індексом 6 ($\normalsize\texttt{array[6]}$) не потребує нового значення оскільики є найбішим значенням сортування цієї ітерації.
  \item Кінець процедури сортування.
\end{enumerate}
\begin{center}
  \Large{\cyr{\textbf{Додатково}}}
\end{center}
\begin{itemize}
  \item В файлі $\footnotesize\texttt{code.cpp}$ подано дві імплементацію алгоритму сортування вставкою ($\normalsize\texttt{array\_insertion\_sort}$ та $\normalsize\texttt{array\_insertion\_sort\_inplace\_swap}$).
  \item Приведено приклад реалізації сортування (який можна скомпілювати та запустити насутпним чином): \\
  $\normalsize\texttt{c++ problem\_2\_sort.cpp code.cpp -o sort -std=c++11 \&\& ./sort}$
\end{itemize}
