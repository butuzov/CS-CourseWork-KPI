% Закони та операції над множинами
%
% Данні закони оформлені на одному листку студентом
% заочної освіти Олегом Бутузовим - в рамках вивчення
% - Програмного пакету LaTeX,
% - Логічних операцій ( у рамках предмету Дискрена математика)
% за допомогою наступних джерел
% - Конспекти лекцій з Дискретної математики Спекторського I.Я.
% - Методички з розрахунків з Дискретної математики Спекторського I.Я.
% - Інтернету

\documentclass[a4paper, 12pt, oneside, titlepage, BCOR=1mm, DIV=12]{scrreprt}
\usepackage[a4paper,top=2cm,bottom=2cm,left=1.5cm,right=1.5cm]{geometry}

% Українська
\usepackage[utf8]{inputenc}
\usepackage[english,ukrainian]{babel}
\usepackage{fancyhdr}
\usepackage{mathtext}
\usepackage{amsmath,amssymb}


% sets
\newcommand{\compl}{^\mathsf{c}}
\let\emptyset\varnothing

% ні номерам сторінок
\pagenumbering{gobble}

% заміна значка пустої підмножини на нормальну
% https://tex.stackexchange.com/questions/22798/nice-looking-empty-set
\usepackage{amssymb}
\let\oldemptyset\emptyset
\let\emptyset\varnothing
\newcommand{\compl}{^\mathsf{c}}

\begin{document}

  \begin{center}\large{\cyr{\textbf{Основні Логічні Тотожності}}}\end{center}

  \begin{enumerate}
    \item \normalsize{\cyr{Закон Комутативності / Commutative Laws}}

      \begin{displaymath}
        A\cup{B} = B\cup{A} \qquad A\cap{B} = B\cap{A}
      \end{displaymath}

    \item \normalsize{\cyr{Закони Розподільності / Дистрибутивність / Distributive Laws}}

      \begin{displaymath}
        A\cup{(B\cap{C})} = (A\cup{B})\cap{(A\cup{C})} \qquad A\cap{(B\cup{C})} = (A\cap{B})\cup{(A\cap{C})}
      \end{displaymath}

    \item \normalsize{\cyr{Закони Нейтральності / Identity Laws / Domination Laws}}

      \begin{displaymath}
        A\cup{\emptyset} = A \qquad A\cap{U}=A
      \end{displaymath}

    \item \normalsize{\cyr{Закони Доповненості / Complement Laws}}

      \begin{displaymath}
        A \cup{A\compl} = U \quad A\cap{A}\compl=\emptyset
      \end{displaymath}

    \item \normalsize{\cyr{Закони Універсальних границь / Universal Bounds }}

      \begin{displaymath}
        A \cup{U} = U \quad A\cap\emptyset=\emptyset
      \end{displaymath}

    \item \normalsize{\cyr{Закони Абсорбції / Absorption Laws }}

      \begin{displaymath}
        A\cup{(A\cap{B})} = A \quad A\cap{(A\cup{B})} = A
      \end{displaymath}

    \item \normalsize{\cyr{ Закони Ідемподентості / Idempodent Laws }}

      \begin{displaymath}
        A\cup{A} = A \quad A\cap{A} = A
      \end{displaymath}

    \item \normalsize{\cyr{Закони Ассоціативності / Associative Laws}}

      \begin{displaymath}
        A\cup{(B\cup{C})}=(A\cup{B})\cup{C} \quad A\cap{(B\cap{C})} = (A\cap{B})\cap{C}
      \end{displaymath}

    \item \normalsize{\cyr{Закони Єдності доповнення}}
      \begin{displaymath}
        \left\{\begin{array}{l}
            A \cup{X} = U \\
            A \cap{X} = \emptyset
            \end{array} \Rightarrow (X = A\compl)
      \end{displaymath}

    \item \normalsize{\cyr{Інволютивність / Involution}}
      \begin{displaymath}
        (A\compl)\compl = A
      \end{displaymath}

    \item \normalsize{\cyr{Закон Де Моргана / De Morgan's Law}}
      \begin{displaymath}
        (A\cup{B})\compl = A\compl\cap{B\compl} \qquad (A\cap{B})\compl = A\compl\cup{B\compl}
      \end{displaymath}

    \item \normalsize{\cyr{Закон Порецького}}
      \begin{displaymath}
        A\cap{(A\compl\cup{B})}=A\cap{B} \qquad A\cup{(A\compl\cap{B})}=A\cup{B}
      \end{displaymath}

    \item \normalsize{\cyr{Закон Склеювання}}
      \begin{displaymath}
        (A\cup{B})\cap{(A\cup{B\compl})} = A \qquad    (A\cap{B})\cup{(A\cap{B\compl})} = A
      \end{displaymath}

  \end{enumerate}
\end{document}
